\documentclass[12pt, a4paper]{article} % Formato de plantlla que vamos a utilizar

\usepackage[utf8]{inputenc}
\usepackage[spanish]{babel}
\usepackage{setspace}
\usepackage[margin=2.5cm, left=3cm, right=2cm, includefoot]{geometry}
\usepackage{graphicx} % Inserción de imágenes
\usepackage[dvipsnames, table, xcdraw]{xcolor}
\usepackage[most]{tcolorbox} % Inserción de cuadros en la portada
\usepackage{fancyhdr} % Definir el estilo de la página
\usepackage[hidelinks]{hyperref} % Gestión de hipervínculos
\usepackage{listings} % Para la inserción de código
\usepackage{parskip} % Arreglo de la tabulación en el documento
\usepackage[figurename=Fig.]{caption} % Cambiar el nombre del caption de las fotos
\usepackage{smartdiagram} % Inserción de Diagramas
\usepackage{zed-csp} % Inserción de esquemas
\usepackage{hyperref} % Para hipervínculos
\usepackage{setspace}
\usepackage{titlesec}
\usepackage{blindtext} % Solo para generar texto de relleno, puedes eliminar esta línea en tu documento final.
\usepackage{natbib}

% Encabezado y pie de página
\pagestyle{fancy}
\fancyhf{}
\renewcommand{\headrulewidth}{0pt} % Elimina la línea del encabezado
%\fancyfoot[C,CO]{\thepage} % Número de página en el centro para páginas pares e impares
%\fancyhead[RO]{\textbf{\theTitle}} % Nombre de la materia en páginas impares
%\fancyhead[L]{\textbf{\theauthor}} % Título de la investigación en páginas pares
\setlength{\headheight}{15pt} % Ajuste necesario para evitar warnings

% Configuración de fuentes
\usepackage{times} % Times New Roman
%\usepackage{tgheros} % Tahoma
%\renewcommand{\familydefault}{\sfdefault} % Descomenta esta línea si usas Tahoma
%\usepackage{uarial} % Arial
%\renewcommand{\rmdefault}{phv} % Descomenta esta línea si usas Arial

% Ajustes de formato
\renewcommand{\baselinestretch}{1.15} % Espaciado de línea anterior
\setlength{\parskip}{9pt} % Espaciado de línea anterior
\setlength{\parindent}{0pt} % Sangría

% Declaración de colores
\definecolor{greenPortada}{HTML}{69A84F}
\definecolor{LightGray}{gray}{0.9}
\definecolor{codegreen}{rgb}{0, 0.6, 0}
\definecolor{codegray}{rgb}{0.5, 0.5, 0.5}
\definecolor{codepurple}{rgb}{0.58, 0, 0.82}
\definecolor{backcolour}{rgb}{0.95, 0.95, 0.92}

% Esilo de enlaces
\hypersetup{
	colorlinks=true,
	linkcolor=greenPortada,
	filecolor=greenPortada,
	urlcolor=greenPortada,
	pdftitle={Overleaf Example},
	pdfpagemode=FullScreen,
}

\urlstyle{same}

% Estilo de bloques de código
\lstdefinestyle{mystyle}{
	backgroundcolor=\color{backcolour},
	commentstyle=\color{codegreen},
	keywordstyle=\color{magenta},
	numberstyle=\tiny\color{codegray},
	stringstyle=\color{codepurple},
	basicstyle=\ttfamily\footnotesize,
	breakatwhitespace=false,
	breaklines=true,
	captionpos=b,
	keepspaces=true,
	numbers=left,
	numbersep=5pt,
	showspaces=false,
	showstringspaces=false,
	showtabs=false,
	tabsize=2
}

\lstset{style=mystyle}

% Declaración de variables
\newcommand{\logoUdg}{../../../../attachments/images/portada-udg.jpeg}
\newcommand{\logoCucei}{../../../../attachments/images/portada-cucei.jpeg}

% Datos de la Materia
\newcommand{\materia}{Análisis de Algoritmos}
\newcommand{\theTitle}{3. Problema de la Suma de Subconjuntos}
\newcommand{\profesor}{Jorgn Ernesto, López Arce Delgado}
\newcommand{\theAuthor}{Alan Yahir Juárez Rubio}
\newcommand{\code}{218517809}
\newcommand{\carrera}{Ingeniería en Cómputo}
\newcommand{\seccion}{D01}
\newcommand{\startDate}{06 de febrero de 2024}

% Espaciado
\newcommand{\nl}{\par\vspace{0.4cm}}

\newcommand{\website}{https://hackthebox.eu} % Sitio Web

% Adicionales
\addto\captionsspanish{\renewcommand{\contentsname}{Índice}}
\setlength{\headheight}{60pt}
\pagestyle{fancy}

% Encabezado
\fancyhf{}
\lhead{
	\begin{minipage}[c][2cm][c]{1.3cm}
		\begin{flushleft}
			\includegraphics[width=5cm, height=1.4cm, keepaspectratio]{\logoUdg}
		\end{flushleft}
	\end{minipage}
	\begin{minipage}[c][2cm][c]{0.45\textwidth} % Adjust the height as needed
		\begin{flushleft}	
		{\ifodd\value{page}\materia\fi}
		\end{flushleft}
	\end{minipage}
}
\rhead{
		\begin{minipage}[c][2cm][c]{0.45\textwidth} % Adjust the height as needed
			\begin{flushright}
				{\ifodd\value{page}\else\theTitle\fi}
			\end{flushright}
		\end{minipage}
		\begin{minipage}[c][2cm][c]{1.3cm}
			\begin{flushright}
				\includegraphics[width=5cm, height=1.4cm, keepaspectratio]{\logoCucei}
			\end{flushright}
		\end{minipage}
}

% Pie de página
\rfoot{\theAuthor}

\renewcommand{\headrulewidth}{3pt}
\renewcommand{\headrule}{\hbox to\headwidth{\color{greenPortada}\leaders\hrule height \headrulewidth\hfill}}

\renewcommand{\lstlistingname}{Código} % Para cambiar el caption de los código
%\renewcommand\thepage{\ifodd\value{page}\else\arabic{page}\fi}

\title{\theTitle}
\author{\theAuthor}

% Comienzo del Documento
\begin{document}
\cfoot{\thepage} % Paginación
\setstretch{1}

\begin{titlepage}
	\centering
	{\huge\textbf{Universidad de Guadalajara}}\par\vspace{0.6cm}
	{\LARGE{Centro Universitario de Ciencias Exactas e Ingenierías}}\vfill
	
	\begin{figure}[h]
		\begin{minipage}[t]{0.45\textwidth}
			\centering
			\includegraphics[width=130px, height=300px, keepaspectratio]{\logoUdg}
		\end{minipage}
		\hfill
		\begin{minipage}[t]{0.45\textwidth}
			\centering
			\includegraphics[width=130px, height=300px, keepaspectratio]{\logoCucei}
		\end{minipage}
	\end{figure}\vfill
	
	{\Large{División de Tecnologías para la Integración CiberHumana}}\vfill
	{\Large\textbf{\materia}}\vfill
	\begin{figure}[h]
		\centering
		\begin{minipage}[t]{0.75\textwidth}
			{\Large
				\textbf{Profesor:} \profesor\nl
				\textbf{Alumno:} \theAuthor\nl
				\textbf{Código:} \code\nl
				\textbf{Carrera:} \carrera\nl
				\textbf{Sección:} \seccion
			}
		\end{minipage}
	\end{figure}\vfill
	{\LARGE{\textbf{\theTitle}}}\vfill
	
	\begin{tcolorbox}[colback=red!5!white, colframe=red!75!black]
		\centering
		Este documento contiene información sensible.\\
		No debería ser impreso o compartido con terceras entidades.
	\end{tcolorbox}\vfill
	{\large \startDate}\par
\end{titlepage}

% ïndices
\clearpage
\tableofcontents

%\clearpage
%\listoffigures
	
%\clearpage
%\listoftables
% Fin de Índices

\clearpage
\section{Introducción}

El problema de la \textbf{suma de subconjuntos} es un problema muy interesante ya que plantea si es posible que de un conjunto de números enteros, es posible encontrar aquellos subconjuntos que sumen un número entero en específico.

Un álgoritmo de fuerza bruta es aquel en el que se prueban todas las combinaciones posibles de un conjunto de datos para, en base a estos, encontrar todos los resultados posibles que dén un resultado determinado.

Para resolver este problema con un algorimto de fuerza bruta basta con generar cada una de las combinaciones posibles (subonjuntos), sumar los subconjuntos y comparar cada el número objetivo.

\section{Objetivos}

En esta activdad se plantea encontrar todas las posibles sumas de subconjuntos de un conjunto que den un reultado en específico implementando un \textbf{algoritmo de fuerza bruta}. 

\subsection{Objetivos Generales}

\begin{itemize}
	\item Genenrar un algoritmo de fuerza bruta que resuelva el problema de la suma de subconjuntos
\end{itemize}

\subsection{Objetivos Específicos}

\begin{itemize}
	\item Documenta los	resultados obtenidos, incluyendo ejemplos de ejecución con
	diferentes conjuntos de datos y valores objetivos.
	\item Analiza la eficacia y limitaciones de este enfoque de fuerza bruta para el
	Problema de la Suma de Subconjuntos.
\end{itemize}

\section{Desarrollo}

Para la solución del problema de esta práctica implementé la función \texttt{subset\_sum} la cual toma como parámetros \texttt{arr} y \texttt{target\_sum}. Esta función utiliza la función auxiliar \texttt{gen\_subsets}.

La función auxiliar \texttt{gen\_subsets} es una función recursiva que la cual genera 3 parámetros:

\begin{itemize}
	\item \texttt{i}: Índice de \texttt{arr}. Ayuda a indicar qué elemento estamos incluyendo o excluyendo del suconjunto actual
	\item \texttt{current\_subsect}: Subconjunto actualmente generado
	\item \texttt{current\_sum}: Suma de los elementos del conjunto generado
\end{itemize}

Para explorar todas las combinaciones posibles primeramente empezamos llamando dicha con valores con 0 y una lista vacia

A esta lista se le va insertando el elemento siguiente y, a su vez, se excluye el elemento actual.

El caso base de esta función recursiva es cuando el índice \texttt{i} es igual al número de elementos de \texttt{arr}. Si se cumple esta condición se aplica un \texttt{return} para finalizar la función recursiva.

Adicionalmente agregué un menú en el cual viene una función para una lista de números aleatorios desede \texttt{min} a \texttt{max} (establecido de 0 a 10), una función que permite guardar números en una lista (generación manual de la lista), una opción que imprime la lista   

\subsection{Comprobación del Funcionamiento del Algoritmo}

\begin{figure}[h]
	\centering
	\includegraphics{img/image.jpg}
\end{figure}

\begin{figure}[h]
	\centering
	\includegraphics{img/image-1.jpg}
\end{figure}

\begin{figure}[h]
	\centering
	\includegraphics{img/image-2.jpg}
\end{figure}

\begin{figure}[h]
	\centering
	\includegraphics{img/image-3.jpg}
\end{figure}


\clearpage

\section{Conclusión}

En retrospectiva, un algoritmo de fuerza bruta es una opción en la que siempre se puede llegar a probar todas las combinaciones de un problema y con ello solucionar un problema; sin embargo, no siempre es la mejor opción. Conforma la cantidad de elementos a combinar sea mayor, la complejidad del algoritmo escala muchísimo, consumiendo mucha memoria y espacio.

Para finalizar, respecto al algoritmo implementado, se pueden hacer varias modificaciones para mejorar en cuestiones de rendimiento: p. ej: bien se podría descartar todos aquellos números que sean mayor que la número objetivo (suma a encontrar); otro punto que se podría atacar es la sumas de subconjuntos de dos elementos: si el número objetivo es par, se podrían descartar aquellas combinaciones que incluyan un número par y uno impar

% Referencias

\nocite{*} % Para incluir todas las referencias sin necesidad de citarlas

\clearpage
\bibliographystyle{apalike}
\bibliography{ref}

\end{document}
