\documentclass[12pt, a4paper]{article} % Formato de plantlla que vamos a utilizar

\usepackage[utf8]{inputenc}
\usepackage[spanish]{babel}
\usepackage{setspace}
\usepackage[margin=2.5cm, left=3cm, right=2cm, includefoot]{geometry}
\usepackage{graphicx} % Inserción de imágenes
\usepackage[dvipsnames, table, xcdraw]{xcolor}
\usepackage[most]{tcolorbox} % Inserción de cuadros en la portada
\usepackage{fancyhdr} % Definir el estilo de la página
\usepackage[hidelinks]{hyperref} % Gestión de hipervínculos
\usepackage{listings} % Para la inserción de código
\usepackage{parskip} % Arreglo de la tabulación en el documento
\usepackage[figurename=Fig.]{caption} % Cambiar el nombre del caption de las fotos
\usepackage{smartdiagram} % Inserción de Diagramas
\usepackage{zed-csp} % Inserción de esquemas
\usepackage{hyperref} % Para hipervínculos
\usepackage{setspace}
\usepackage{titlesec}
\usepackage{blindtext} % Solo para generar texto de relleno, puedes eliminar esta línea en tu documento final.
\usepackage{natbib}

% Encabezado y pie de página
\pagestyle{fancy}
\fancyhf{}
\renewcommand{\headrulewidth}{0pt} % Elimina la línea del encabezado
%\fancyfoot[C,CO]{\thepage} % Número de página en el centro para páginas pares e impares
%\fancyhead[RO]{\textbf{\theTitle}} % Nombre de la materia en páginas impares
%\fancyhead[L]{\textbf{\theauthor}} % Título de la investigación en páginas pares
\setlength{\headheight}{15pt} % Ajuste necesario para evitar warnings

% Configuración de fuentes
\usepackage{times} % Times New Roman
%\usepackage{tgheros} % Tahoma
%\renewcommand{\familydefault}{\sfdefault} % Descomenta esta línea si usas Tahoma
%\usepackage{uarial} % Arial
%\renewcommand{\rmdefault}{phv} % Descomenta esta línea si usas Arial

% Ajustes de formato
\renewcommand{\baselinestretch}{1.15} % Espaciado de línea anterior
\setlength{\parskip}{9pt} % Espaciado de línea anterior
\setlength{\parindent}{0pt} % Sangría

% Declaración de colores
\definecolor{greenPortada}{HTML}{69A84F}
\definecolor{LightGray}{gray}{0.9}
\definecolor{codegreen}{rgb}{0, 0.6, 0}
\definecolor{codegray}{rgb}{0.5, 0.5, 0.5}
\definecolor{codepurple}{rgb}{0.58, 0, 0.82}
\definecolor{backcolour}{rgb}{0.95, 0.95, 0.92}

% Esilo de enlaces
\hypersetup{
	colorlinks=true,
	linkcolor=greenPortada,
	filecolor=greenPortada,
	urlcolor=greenPortada,
	pdftitle={Overleaf Example},
	pdfpagemode=FullScreen,
}

\urlstyle{same}

% Estilo de bloques de código
\lstdefinestyle{mystyle}{
	backgroundcolor=\color{backcolour},
	commentstyle=\color{codegreen},
	keywordstyle=\color{magenta},
	numberstyle=\tiny\color{codegray},
	stringstyle=\color{codepurple},
	basicstyle=\ttfamily\footnotesize,
	breakatwhitespace=false,
	breaklines=true,
	captionpos=b,
	keepspaces=true,
	numbers=left,
	numbersep=5pt,
	showspaces=false,
	showstringspaces=false,
	showtabs=false,
	tabsize=2
}

\lstset{style=mystyle}

% Declaración de variables
\newcommand{\logoUdg}{../../../../attachments/images/portada-udg.jpeg}
\newcommand{\logoCucei}{../../../../attachments/images/portada-cucei.jpeg}

% Datos de la Materia
\newcommand{\materia}{Análisis de Algoritmos}
\newcommand{\theTitle}{5. Mediana de un Arreglo: Aplicación del Paradigma Divide y Vencerás}
\newcommand{\profesor}{Jorgn Ernesto, López Arce Delgado}
\newcommand{\theAuthor}{Alan Yahir Juárez Rubio}
\newcommand{\code}{218517809}
\newcommand{\carrera}{Ingeniería en Cómputo}
\newcommand{\seccion}{D01}
\newcommand{\startDate}{29 de febrero de 2024}

% Espaciado
\newcommand{\nl}{\par\vspace{0.4cm}}

\newcommand{\website}{https://hackthebox.eu} % Sitio Web

% Adicionales
\addto\captionsspanish{\renewcommand{\contentsname}{Índice}}
\setlength{\headheight}{60pt}
\pagestyle{fancy}

% Encabezado
\fancyhf{}
\lhead{
	\begin{minipage}[c][2cm][c]{1.3cm}
		\begin{flushleft}
			\includegraphics[width=5cm, height=1.4cm, keepaspectratio]{\logoUdg}
		\end{flushleft}
	\end{minipage}
	\begin{minipage}[c][2cm][c]{0.45\textwidth} % Adjust the height as needed
		\begin{flushleft}	
		{\ifodd\value{page}\materia\fi}
		\end{flushleft}
	\end{minipage}
}
\rhead{
		\begin{minipage}[c][2cm][c]{0.45\textwidth} % Adjust the height as needed
			\begin{flushright}
				{\ifodd\value{page}\else\theTitle\fi}
			\end{flushright}
		\end{minipage}
		\begin{minipage}[c][2cm][c]{1.3cm}
			\begin{flushright}
				\includegraphics[width=5cm, height=1.4cm, keepaspectratio]{\logoCucei}
			\end{flushright}
		\end{minipage}
}

% Pie de página
\rfoot{\theAuthor}

\renewcommand{\headrulewidth}{3pt}
\renewcommand{\headrule}{\hbox to\headwidth{\color{greenPortada}\leaders\hrule height \headrulewidth\hfill}}

\renewcommand{\lstlistingname}{Código} % Para cambiar el caption de los código
%\renewcommand\thepage{\ifodd\value{page}\else\arabic{page}\fi}

\title{\theTitle}
\author{\theAuthor}

% Comienzo del Documento
\begin{document}
\cfoot{\thepage} % Paginación
\setstretch{1}

\begin{titlepage}
	\centering
	{\huge\textbf{Universidad de Guadalajara}}\par\vspace{0.6cm}
	{\LARGE{Centro Universitario de Ciencias Exactas e Ingenierías}}\vfill
	
	\begin{figure}[h]
		\begin{minipage}[t]{0.45\textwidth}
			\centering
			\includegraphics[width=130px, height=300px, keepaspectratio]{\logoUdg}
		\end{minipage}
		\hfill
		\begin{minipage}[t]{0.45\textwidth}
			\centering
			\includegraphics[width=130px, height=300px, keepaspectratio]{\logoCucei}
		\end{minipage}
	\end{figure}\vfill
	
	{\Large{División de Tecnologías para la Integración CiberHumana}}\vfill
	{\Large\textbf{\materia}}\vfill
	\begin{figure}[h]
		\centering
		\begin{minipage}[t]{0.75\textwidth}
			{\Large
				\textbf{Profesor:} \profesor\nl
				\textbf{Alumno:} \theAuthor\nl
				\textbf{Código:} \code\nl
				\textbf{Carrera:} \carrera\nl
				\textbf{Sección:} \seccion
			}
		\end{minipage}
	\end{figure}\vfill
	{\LARGE{\textbf{\theTitle}}}\vfill
	
	\begin{tcolorbox}[colback=red!5!white, colframe=red!75!black]
		\centering
		Este documento contiene información sensible.\\
		No debería ser impreso o compartido con terceras entidades.
	\end{tcolorbox}\vfill
	{\large \startDate}\par
\end{titlepage}

% ïndices
\clearpage
\tableofcontents

\clearpage
\listoffigures
	
%\clearpage
%\listoftables
% Fin de Índices

\clearpage
\section{Introducción}

El problema de la \textbf{mediana de un arreglo} es un problema muy común y muy fácil de resolver, sin embargo, al manejar un gran conjunto de datos, este puede llegar a ser algo ineficiente.

Una de las alternativas para solucionar este problema con mayor eficiencia es utilizando el paradigma de divide y vencerás. Este paradigma, a grandes rasgos, consiste en dividir un conjunto de datos en subconjuntos para trabajar/resolver cada subconjunto para al combinar cada una de las soluciones para conseguir la solución al problema principal.

\subsection{Objetivos Generales}

\begin{itemize}
	\item Desarrollar un algoritmo que obtenga la mediana de un arreglo aplicando el paradigma de \textbf{divide y vencerás} en base al video proporcionado.
\end{itemize}

\subsection{Objetivos Específicos}

\begin{itemize}
	\item Documenta los	resultados obtenidos, incluyendo ejemplos de ejecución con
	diferentes conjuntos de datos y valores objetivos.
\end{itemize}

\section{Desarrollo}

Para la solución del problema de esta práctica desarrollé 3 funciones:

\begin{itemize}
	\item \texttt{gen\_random(arr, min, max)}: Dado un arreglo, un mínimo y un máximo, agrega al arreglo $n$ cantidad de números enteros aleatorios dentro del rango [\texttt{min}, \texttt{max}].
	\item \texttt{median\_of\_medians(A, i)}: Dado un arreglo y un índice, Te regresa el valor correspondiente al valor del arreglo ordenado en dicho índice.
	\item \texttt{print\_median\_arr(arr)}; Dado un arreglo, imprime la lista y, haciendo uso de la función \texttt{median\_of\_medians(A, i)}, imprime la media de la lista. Este se encarga de hacer funcionar dicha función tanto para listas de tamaño impar como de tamaño par.
\end{itemize}

\clearpage
\subsection{Comprobación del Funcionamiento del Algoritmo}

\begin{figure}[h]
	\centering
	\includegraphics{img/image.jpg}
	\caption{Comprobación del Algoritmo con 4 arreglos aleatorios de diferentes tamaños}
\end{figure}

\clearpage
\section{Conclusión}

En retrospectiva, un algoritmo de divide y vencerás es una alternativa muy óptima en términos de eficiencia. Como se previamente se mencionó, este se encarga de dividir el problema en subproblemas más pequeños para al final fusionar la solución de los subproblemas y así obtener la solución del problema principal.

En conclusión, el estudio y desarrollo de este algoritmo me fue de mucha utilidad debido a que aprendí a cómo implementar el paradigma de dividir y conquistar. Este problema en concreto se me hizo muy interesante debido que, en base a un problema que la mayoría sabe resolver, se puede generar un algoritmo que, para un computador es mucho más eficiente resolver. Si bien, este algoritmo es un tanto más complejo y difícil de entender, en términos de eficiencia computacional es mucho mejor.

% Referencias

\nocite{*} % Para incluir todas las referencias sin necesidad de citarlas

\clearpage
\bibliographystyle{apalike}
\bibliography{ref}

\end{document}
