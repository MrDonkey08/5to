\documentclass[12pt, a4paper]{article} % Formato de plantlla que vamos a utilizar

\usepackage[utf8]{inputenc}
\usepackage[spanish]{babel}
\usepackage{setspace}
\usepackage[margin=2.5cm, left=3cm, right=2cm, includefoot]{geometry}
\usepackage{graphicx} % Inserción de imágenes
\usepackage[dvipsnames, table, xcdraw]{xcolor}
\usepackage[most]{tcolorbox} % Inserción de cuadros en la portada
\usepackage{fancyhdr} % Definir el estilo de la página
\usepackage[hidelinks]{hyperref} % Gestión de hipervínculos
\usepackage{listings} % Para la inserción de código
\usepackage{parskip} % Arreglo de la tabulación en el documento
\usepackage[figurename=Fig.]{caption} % Cambiar el nombre del caption de las fotos
\usepackage{smartdiagram} % Inserción de Diagramas
\usepackage{zed-csp} % Inserción de esquemas
\usepackage{hyperref} % Para hipervínculos
\usepackage{setspace}
\usepackage{titlesec}
\usepackage{blindtext} % Solo para generar texto de relleno, puedes eliminar esta línea en tu documento final.
\usepackage{natbib}

% Encabezado y pie de página
\pagestyle{fancy}
\fancyhf{}
\renewcommand{\headrulewidth}{0pt} % Elimina la línea del encabezado
%\fancyfoot[C,CO]{\thepage} % Número de página en el centro para páginas pares e impares
%\fancyhead[RO]{\textbf{\theTitle}} % Nombre de la materia en páginas impares
%\fancyhead[L]{\textbf{\theauthor}} % Título de la investigación en páginas pares
\setlength{\headheight}{15pt} % Ajuste necesario para evitar warnings

% Configuración de fuentes
\usepackage{times} % Times New Roman
%\usepackage{tgheros} % Tahoma
%\renewcommand{\familydefault}{\sfdefault} % Descomenta esta línea si usas Tahoma
%\usepackage{uarial} % Arial
%\renewcommand{\rmdefault}{phv} % Descomenta esta línea si usas Arial

% Ajustes de formato
\renewcommand{\baselinestretch}{1.15} % Espaciado de línea anterior
\setlength{\parskip}{9pt} % Espaciado de línea anterior
\setlength{\parindent}{0pt} % Sangría

% Declaración de colores
\definecolor{greenPortada}{HTML}{69A84F}
\definecolor{LightGray}{gray}{0.9}
\definecolor{codegreen}{rgb}{0, 0.6, 0}
\definecolor{codegray}{rgb}{0.5, 0.5, 0.5}
\definecolor{codepurple}{rgb}{0.58, 0, 0.82}
\definecolor{backcolour}{rgb}{0.95, 0.95, 0.92}

% Esilo de enlaces
\hypersetup{
	colorlinks=true,
	linkcolor=greenPortada,
	filecolor=greenPortada,
	urlcolor=greenPortada,
	pdftitle={Overleaf Example},
	pdfpagemode=FullScreen,
}

\urlstyle{same}

% Estilo de bloques de código
\lstdefinestyle{mystyle}{
	backgroundcolor=\color{backcolour},
	commentstyle=\color{codegreen},
	keywordstyle=\color{magenta},
	numberstyle=\tiny\color{codegray},
	stringstyle=\color{codepurple},
	basicstyle=\ttfamily\footnotesize,
	breakatwhitespace=false,
	breaklines=true,
	captionpos=b,
	keepspaces=true,
	numbers=left,
	numbersep=5pt,
	showspaces=false,
	showstringspaces=false,
	showtabs=false,
	tabsize=2
}

\lstset{style=mystyle}

% Declaración de variables
\newcommand{\logoUdg}{../../../../attachments/images/portada-udg.jpeg}
\newcommand{\logoCucei}{../../../../attachments/images/portada-cucei.jpeg}

% Datos de la Materia
\newcommand{\materia}{Análisis de Algoritmos}
\newcommand{\theTitle}{6. Ensamblado de Secuencias Biológicas: Aplicación del Paradigma Divide y Vencerás}
\newcommand{\profesor}{Jorge Ernesto López Arce Delgado}
\newcommand{\theAuthor}{Alan Yahir Juárez Rubio}
\newcommand{\code}{218517809}
\newcommand{\carrera}{Ingeniería en Cómputo}
\newcommand{\seccion}{D01}
\newcommand{\startDate}{07 de marzo de 2024}

% Espaciado
\newcommand{\nl}{\par\vspace{0.4cm}}

% Adicionales
\addto\captionsspanish{\renewcommand{\contentsname}{Índice}}
\setlength{\headheight}{60pt}
\pagestyle{fancy}

% Encabezado
\fancyhf{}
\lhead{
	\begin{minipage}[c][2cm][c]{1.3cm}
		\begin{flushleft}
			\includegraphics[width=5cm, height=1.4cm, keepaspectratio]{\logoUdg}
		\end{flushleft}
	\end{minipage}
	\begin{minipage}[c][2cm][c]{0.45\textwidth} % Adjust the height as needed
		\begin{flushleft}	
		{\ifodd\value{page}\materia\fi}
		\end{flushleft}
	\end{minipage}
}
\rhead{
		\begin{minipage}[c][2cm][c]{0.45\textwidth} % Adjust the height as needed
			\begin{flushright}
				{\ifodd\value{page}\else\theTitle\fi}
			\end{flushright}
		\end{minipage}
		\begin{minipage}[c][2cm][c]{1.3cm}
			\begin{flushright}
				\includegraphics[width=5cm, height=1.4cm, keepaspectratio]{\logoCucei}
			\end{flushright}
		\end{minipage}
}

% Pie de página
\rfoot{\theAuthor}

\renewcommand{\headrulewidth}{3pt}
\renewcommand{\headrule}{\hbox to\headwidth{\color{greenPortada}\leaders\hrule height \headrulewidth\hfill}}

\renewcommand{\lstlistingname}{Código} % Para cambiar el caption de los código
%\renewcommand\thepage{\ifodd\value{page}\else\arabic{page}\fi}

\title{\theTitle}
\author{\theAuthor}

% Comienzo del Documento
\begin{document}
\cfoot{\thepage} % Paginación
\setstretch{1}

\begin{titlepage}
	\centering
	{\huge\textbf{Universidad de Guadalajara}}\par\vspace{0.6cm}
	{\LARGE{Centro Universitario de Ciencias Exactas e Ingenierías}}\vfill
	
	\begin{figure}[h]
		\begin{minipage}[t]{0.45\textwidth}
			\centering
			\includegraphics[width=130px, height=300px, keepaspectratio]{\logoUdg}
		\end{minipage}
		\hfill
		\begin{minipage}[t]{0.45\textwidth}
			\centering
			\includegraphics[width=130px, height=300px, keepaspectratio]{\logoCucei}
		\end{minipage}
	\end{figure}\vfill
	
	{\Large{División de Tecnologías para la Integración CiberHumana}}\vfill
	{\Large\textbf{\materia}}\vfill
	\begin{figure}[h]
		\centering
		\begin{minipage}[t]{0.75\textwidth}
			{\Large
				\textbf{Profesor:} \profesor\nl
				\textbf{Alumno:} \theAuthor\nl
				\textbf{Código:} \code\nl
				\textbf{Carrera:} \carrera\nl
				\textbf{Sección:} \seccion
			}
		\end{minipage}
	\end{figure}\vfill
	{\LARGE{\textbf{\theTitle}}}\vfill
	
	\begin{tcolorbox}[colback=red!5!white, colframe=red!75!black]
		\centering
		Este documento contiene información sensible.\\
		No debería ser impreso o compartido con terceras entidades.
	\end{tcolorbox}\vfill
	{\large \startDate}\par
\end{titlepage}

% ïndices
\clearpage
\tableofcontents

\clearpage
\listoffigures
	
%\clearpage
%\listoftables
% Fin de Índices

\clearpage
\section{Introducción}

El ensamblado de secuencias biológicas a partir de fragmentos es un problema un tanto particular e interesante. Si bien se puede aplicar fuerza bruta, no es lo más óptimo y eficiente para hacerlo.

Con el paradigma de divide y vencerás, nos perimite evaluar cada uno de los fragmentos para determinar qué elementos se interpolan con el fragmento anterior. De esta manera se puede reconstruir en base a los fragmentos el secuencia biológica original.

\subsection{Objetivos Generales}

\begin{itemize}
	\item Aprender a implementar el paradigma de divide y vencerás
\end{itemize}

\subsection{Objetivos Específicos}

\begin{itemize}
	\item Dado un csv, armar una secuencia biologica a partir de los fragmentos obtenidos utilizando el paradigma de divide y vencerás.
\end{itemize}

\section{Desarrollo}

Para el desarrollo de esta práctica implementé 4 funciones

\begin{itemize}
	\item \texttt{read()}: Utilizando \texttt{tkinter}, permite seleccionar el archivo CSV con el que se trabajará.
	\item \texttt{get\_fragments(file)}: Del CSV obtiene y almacena cada uno de los \textbf{fragmentos} en un arreglo de strings.
	\item \texttt{get\_string(file)}: Obtiene la \textbf{secuencia objetivo} y la almacena en un string.
	\item \texttt{merge\_fragments(string\_1, string\_2)}: Dado dos strings, los compara y regresa el valor no interpolado del \texttt{string\_2} respecto al \texttt{string\_1}
\end{itemize}

Esta última función lo que básicamente hace es iterar sobre cada uno de los substrings del \texttt{string\_2}. Si el final del \tetxttt{string\_1} coincide con dicho subarreglo, regresa el resto del arreglo, es decir, el valor no interpolado del string.

Si bien, esta función solo sirve para dos strings. En la implementación, haciendo uso de un \texttt{for}, pude comparar cada una de los fragmentos con su anterior (a excepción del primero) y fue guardando cada uno de los valores no interpolados

\begin{tcolorbox}[colback=yellow!10!white,colframe=red!75!black,title=Importante]
	Como en la secuencia biológica del csv proporcionado tiene dos carácteres adicionales, no es posible generar reconstruir en su totalidad el string original. Por lo tanto, decidí verificar que el string formado se encuentra dentro del string original. 
\end{tcolorbox}

\subsection{Implementación del Algoritmo}

\begin{figure}[h]
	\centering
	\includegraphics[width=0.8\textwidth]{img/image.jpg}
	\caption{Selección del archivo CSV}
\end{figure}

\begin{figure}[h]
	\centering
	\includegraphics[width=0.8\textwidth]{img/image_1.jpg}
	\caption{Resultados de la implementación}
\end{figure}

\begin{figure}[h]
	\centering
	\includegraphics[width=0.8\textwidth]{img/image_2.jpg}
	\caption{Últimos carácteres de cada una de las cadenas}
\end{figure}

\clearpage
\section{Conclusión}

En retrospectiva, un algoritmo de divide y vencerás es una alternativa muy óptima en términos de eficiencia. Como se previamente se mencionó, este se encarga de dividir el problema en subproblemas más pequeños para al final fusionar la solución de los subproblemas y así obtener la solución del problema principal.

En conclusión, el estudio y desarrollo de este algoritmo me fue de mucha utilidad debido a que aprendí a cómo implementar el paradigma de dividir y conquistar. Si bien, el algoritmo desarrollado es un tanto sencillo, fue muy difícil de implementar debido a que requirio de varios análisis para poder idear un algoritmo que funcionara correctamente y que utilizara el paradigma de divide y vencerás
% Referencias

\nocite{*} % Para incluir todas las referencias sin necesidad de citarlas

\clearpage
\bibliographystyle{apalike}
\bibliography{ref}

\end{document}
