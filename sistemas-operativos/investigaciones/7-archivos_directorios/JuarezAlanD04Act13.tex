\documentclass[12pt, a4paper]{article} % Formato de plantlla que vamos a utilizar

\usepackage[utf8]{inputenc}
\usepackage[spanish]{babel}
\usepackage{setspace}
\usepackage[margin=2.5cm, left=3cm, right=2cm, includefoot]{geometry}
\usepackage{graphicx} % Inserción de imágenes
\usepackage[dvipsnames, table, xcdraw]{xcolor}
\usepackage[most]{tcolorbox} % Inserción de cuadros en la portada
\usepackage{fancyhdr} % Definir el estilo de la página
\usepackage[hidelinks]{hyperref} % Gestión de hipervínculos
\usepackage{listings} % Para la inserción de código
\usepackage{parskip} % Arreglo de la tabulación en el documento
\usepackage[figurename=Fig.]{caption} % Cambiar el nombre del caption de las fotos
\usepackage{smartdiagram} % Inserción de Diagramas
\usepackage{zed-csp} % Inserción de esquemas
\usepackage{hyperref} % Para hipervínculos
\usepackage{setspace}
\usepackage{titlesec}
\usepackage{blindtext} % Solo para generar texto de relleno, puedes eliminar esta línea en tu documento final.
\usepackage{natbib}
\usepackage{array} % <-- Add this line for m{} column type

% Encabezado y pie de página
\pagestyle{fancy}
\fancyhf{}
\renewcommand{\headrulewidth}{0pt} % Elimina la línea del encabezado
\setlength{\headheight}{15pt} % Ajuste necesario para evitar warnings

% Configuración de fuentes
\usepackage{times} % Times New Roman

% Ajustes de formato
\renewcommand{\baselinestretch}{1.15} % Espaciado de línea anterior
\setlength{\parskip}{6pt} % Espaciado de línea anterior
\setlength{\parindent}{0pt} % Sangría

% Declaración de colores
\definecolor{greenPortada}{HTML}{69A84F}
\definecolor{LightGray}{gray}{0.9}
\definecolor{codegreen}{rgb}{0, 0.6, 0}
\definecolor{codegray}{rgb}{0.5, 0.5, 0.5}
\definecolor{codepurple}{rgb}{0.58, 0, 0.82}
\definecolor{backcolour}{rgb}{0.95, 0.95, 0.92}

% Esilo de enlaces
\hypersetup{
	colorlinks=true,
	linkcolor=greenPortada,
	filecolor=greenPortada,
	urlcolor=greenPortada,
	pdftitle={Overleaf Example},
	pdfpagemode=FullScreen,
}

\urlstyle{same}

% Estilo de bloques de código
\lstdefinestyle{mystyle}{
	backgroundcolor=\color{backcolour},
	commentstyle=\color{codegreen},
	keywordstyle=\color{magenta},
	numberstyle=\tiny\color{codegray},
	stringstyle=\color{codepurple},
	basicstyle=\ttfamily\footnotesize,
	breakatwhitespace=false,
	breaklines=true,
	captionpos=b,
	keepspaces=true,
	numbers=left,
	numbersep=5pt,
	showspaces=false,
	showstringspaces=false,
	showtabs=false,
	tabsize=2
}

\lstset{style=mystyle}

% Declaración de variables
\newcommand{\logoUdg}{../../../../attachments/images/portada-udg.jpeg}
\newcommand{\logoCucei}{../../../../attachments/images/portada-cucei.jpeg}

% Datos de la Materia
\newcommand{\materia}{Sistemas Operativos}
\newcommand{\theTitle}{7. Archivos y Directorios}
\newcommand{\profesor}{Violeta del Rocío Becerra Velázquez}
\newcommand{\theAuthor}{Alan Yahir Juárez Rubio}
\newcommand{\code}{218517809}
\newcommand{\carrera}{Ingeniería en Cómputo}
\newcommand{\seccion}{D04}
\newcommand{\startDate}{28 de abril de 2024}

% Espaciado
\newcommand{\nl}{\par\vspace{0.4cm}}

% Adicionales
\addto\captionsspanish{\renewcommand{\contentsname}{Índice}}
\setlength{\headheight}{60pt}
\pagestyle{fancy}

% Encabezado
\fancyhf{}
\lhead{
	\begin{minipage}[c][2cm][c]{1.3cm}
		\begin{flushleft}
			\includegraphics[width=5cm, height=1.4cm, keepaspectratio]{\logoUdg}
		\end{flushleft}
	\end{minipage}
	\begin{minipage}[c][2cm][c]{0.5\textwidth} % Adjust the height as needed
		\begin{flushleft}	
		{\ifodd\value{page}\materia\fi}
		\end{flushleft}
	\end{minipage}
}
\rhead{
		\begin{minipage}[c][2cm][c]{0.4\textwidth} % Adjust the height as needed
			\begin{flushright}
				{\ifodd\value{page}\else\theTitle\fi}
			\end{flushright}
		\end{minipage}
		\begin{minipage}[c][2cm][c]{1.3cm}
			\begin{flushright}
				\includegraphics[width=5cm, height=1.4cm, keepaspectratio]{\logoCucei}
			\end{flushright}
		\end{minipage}
}

% Pie de página
\rfoot{\ifodd\value{page}\theAuthor\fi}

\renewcommand{\headrulewidth}{3pt}
\renewcommand{\headrule}{\hbox to\headwidth{\color{greenPortada}\leaders\hrule height \headrulewidth\hfill}}

\renewcommand{\lstlistingname}{Código} % Para cambiar el caption de los código
%\renewcommand\thepage{\ifodd\value{page}\else\arabic{page}\fi}

\title{\theTitle}
\author{\theAuthor}

% Comienzo del Documento
\begin{document}
\cfoot{\ifodd\value{page}\else\thepage\fi} % Paginación
\setstretch{1}

\begin{titlepage}
	\centering
	{\huge\textbf{Universidad de Guadalajara}}\par\vspace{0.6cm}
	{\LARGE{Centro Universitario de Ciencias Exactas e Ingenierías}}\vfill
	
	\begin{figure}[h]
		\begin{minipage}[t]{0.45\textwidth}
			\centering
			\includegraphics[width=130px, height=300px, keepaspectratio]{\logoUdg}
		\end{minipage}
		\hfill
		\begin{minipage}[t]{0.45\textwidth}
			\centering
			\includegraphics[width=130px, height=300px, keepaspectratio]{\logoCucei}
		\end{minipage}
	\end{figure}\vfill
	
	{\Large{División de Tecnologías para la Integración CiberHumana}}\vfill
	{\Large\textbf{\materia}}\vfill
	\begin{figure}[h]
		\centering
		\begin{minipage}[t]{0.75\textwidth}
			{\Large
				\textbf{Profesor:} \profesor\nl
				\textbf{Alumno:} \theAuthor\nl
				\textbf{Código:} \code\nl
				\textbf{Carrera:} \carrera\nl
				\textbf{Sección:} \seccion
			}
		\end{minipage}
	\end{figure}\vfill
	{\LARGE{\textbf{\theTitle}}}\vfill
	
	\begin{tcolorbox}[colback=red!5!white, colframe=red!75!black]
		\centering
		Este documento contiene información sensible.\\
		No debería ser impreso o compartido con terceras entidades.
	\end{tcolorbox}\vfill
	{\large \startDate}\par
\end{titlepage}

% ïndices
\clearpage
\tableofcontents

\clearpage
\listoffigures
	
%\clearpage
%\listoftables
% Fin de Índices

\clearpage
\section{Archivos y Directorios}

\subsection{Sistema de Archivos}

Un \textbf{sistema de archivos} es aquel sistema que el SO utiliza para el manejo de archivos: como están estructurados, como son nombrados, accedidos, usados, protegidos, implementados y manejados.

\subsection{Archivos}

Los \textbf{archivos} son un mecanismo de abstracción que proporcionan una forma de almacenar información en dispositivos de almacenamiento secundario.

Cada uno de los diferentes sistemas de archivos provee diferentes operaciones. Entre las más comúnes se encuentran:

\begin{enumerate}
	\item \textbf{\textit{Create}}: Se genera un archivo vacio (sin datos) y le son asignados algunos atributos.
	\item \textbf{\textit{Delete}}: Cuando el archivo ya no es necesario, tiene que ser eliminado para liberar memoria.
	\item \textbf{\textit{Open}}: Antes de utilizar un archivo, un proceso debe abrirlo. El propósito de la llamada \textit{open} es permitir que el sistema recupere los atributos y la lista de las direcciones del disco en memoria principal para rápido acceso en llamadas próximas.
	\item \textbf{\textit{Close}}: Cuando todos los accesos han finalizado, los atributos y las direcciones del disco ya no son necesitadas, por lo cual el archivo debería ser cerrado para liberar espacio en la tabla interna
	\item \textbf{\textit{Read}}: Los datos son leídos del archivo. Usualmente, los bytes vienen de la posición actual. El llamador debe especificar cuántos datos son necesitados y además debe proveer un buffer para introducirlos.
	\item \textbf{\textit{Write}}: Los datos son escritos en el archivo de nuevo, usualmente en la posición actual. Si la posición actual es al final del archivo, el tamaño del archivo incrementa. Si la posición actual está en la mitad del archivo. Los datos existentes son sobrescritos y perdidos por siempre.
	\item \textbf{\textit{Append}}: Esta llamada es una forma retringida de \textit{write}. Añade información solo al final del archivo.
	\item \textbf{\textit{Seek}}: Para el acceso aleatorio de archivos, un método es necesario para especificar de dónde tomar la información. Un enfoque común es la llamada al sistema \textit{seek}, este reposiciona el puntero del archivo a un lugar especifico en el archivo. Después de que la llamda es completada, la información puede ser leída o escrita en esa posición.
	\item \textbf{\textit{Get attributes}}: Los procesos seguidamente necesitan leer los atributos del archivo para realizar su trabajo.
	\item \textbf{\textit{Set attributes}}: Algunos de los atributos pueden ser establecidos por el usuario y pueden ser modificados después de que el archivo ha sido creado. Esta llamada al sistema hace esto posible.
	\item \textbf{\textit{Rename}}: Esta llamada al sistema permite modificar el nombre del archivo. No es estrictamente necesaria porque es posible crear una copia del archivo usando otro nombre y eliminar el archivo viejo.
\end{enumerate}

\subsection{Directorios}

Los directorios son archivos que contienen información sobre otros archivos y se
diferencian de ellos en el modo de acceso.

Las operaciones más comunes entre los sistemas de archivos para los directorios son:

\begin{enumerate}
	\item \textbf{\textit{Create}}: Un directorio nuevo es creado. Está vacio a excepción del punto y del doble punto los cuales son generados automaticamente por el sistema.
	\item \textbf{\textit{Delete}}: Un directorio es eliminado. Solo puede ser eliminado si está vacio, es decir, solo contener punto y doble punto.
	\item \textbf{\textit{Opendir}}: Los directorios pueden ser leídos. Por ejemplo, al listar los archivos de un directorio.
	\item \textbf{\textit{Closedir}}: Cuando un directorio ha sido leído, debería ser cerrado para liberar memoria en la tabla interna.
	\item \textbf{\textit{Readdir}}: Esta llamda regresa la siguiente entrada en un directorio abierto.
	\item \textbf{\textit{Rename}}: En lo que respecta, los directorios son practicamente un archivo. Pueden ser renombrados de la misma manera que los archivos.
	\item \textbf{\textit{Link}}: \textit{Linking} es una técnica que permite a un archivo aparecer en más de un directorio. Esta llamada del sistema especifica un archivo existente y un nombre de una ruta, y crea un enlace desde el archivo existente hasta el nombre de la ruta especificada. Esta implementación es alguna veces llamado como \textbf{\textit{hard link}} (enlace fuerte).
	\item \textbf{\textit{Unlink}}: Una entrada de directorio es eliminada. Si el archivo desenlazado está presente en un directorio (el caso normal), es removido del sistema del archivos. Si está presente en múltiple directorios, entonces solo el nombre del archivo especificado es eliminado. Los otros permanecen.
\end{enumerate}

\clearpage
\section{Almacenamiento y Gestión de Archivos}

\subsection{Índices}

Los \textbf{índices} en SO son estructuras de datos esenciales para optimizar la búsqueda y acceso a información en el sistema de archivos. Su función principal es mejorar el rendimiento de las operaciones de lectura y escritura al permitir la ubicación rápida de archivos y bloques de datos en el almacenamiento. Hay varios tipos de índices utilizados en sistemas operativos:

\begin{enumerate}

	\item \textbf{Índices de Archivos}: Estos índices se utilizan para acceder rápidamente a registros dentro de archivos, permitiendo la búsqueda basada en claves. Los índices pueden ser primarios (basados en la clave principal del registro) o secundarios (basados en una clave no única).
	\item \textbf{Índices de Disco}: Estos índices se utilizan para mapear ubicaciones físicas en el disco con ubicaciones lógicas de archivos y datos. Ayudan a reducir el tiempo de acceso al disco al proporcionar un mapeo eficiente de las estructuras de datos almacenadas en el disco.
	\item \textbf{Índices de Memoria}: Estos índices se utilizan en la gestión de memoria virtual para asignar direcciones virtuales a direcciones físicas, permitiendo el acceso eficiente a datos almacenados en memoria.
\end{enumerate}

Los índices se implementan utilizando diversas estructuras de datos, como tablas de búsqueda, árboles B, tablas hash y estructuras de índices invertidos, dependiendo de los requisitos específicos del sistema operativo y las aplicaciones.

\subsection{Dispersión}

La \textbf{dispersión}, también conocida como hashing, es una técnica fundamental en SO para distribuir elementos en una estructura de datos de manera uniforme. Su objetivo principal es minimizar las colisiones y mejorar el rendimiento de las operaciones de búsqueda, inserción y eliminación. Algunos aspectos importantes de la dispersión en SO son:

\begin{enumerate}
	\item \textbf{Función de Dispersión}: Una función de dispersión convierte claves o datos en una posición dentro de una estructura de datos dispersa, como una tabla hash. Una buena función de dispersión distribuye los elementos de manera uniforme para reducir las colisiones.
	\item \textbf{Tabla Hash}: Es una estructura de datos que utiliza dispersión para almacenar y recuperar datos de manera eficiente. La tabla hash asigna claves a posiciones de almacenamiento utilizando una función de dispersión.
	\item \textbf{Colisiones}: Las colisiones ocurren cuando dos elementos se asignan a la misma posición en la tabla hash debido a la función de dispersión. La resolución de colisiones es un aspecto crítico de la dispersión y se logra mediante métodos como el encadenamiento, la dispersión abierta y la reasignación dinámica.
\end{enumerate}

\subsubsection{Métodos de Dispersion}

En sistemas operativos, existen varios métodos de dispersión utilizados para calcular posiciones alternativas cuando se producen colisiones en una estructura de datos dispersa, como una tabla hash. Algunos métodos comunes de dispersión son:

\begin{enumerate}
	\item \textbf{Dispersión Directa}: También conocida como hashing directo, este método utiliza una función de dispersión simple para calcular la posición de un elemento directamente en la tabla hash.
	\item \textbf{Dispersión Cuadrática}: Este método ajusta la posición de un elemento de manera cuadrática para evitar agrupaciones y mejorar la distribución de elementos en la tabla hash.
	\item \textbf{Dispersión Doble}: Conocida como hashing doble, este método utiliza dos funciones de dispersión para calcular nuevas posiciones en caso de colisiones, lo que reduce la probabilidad de colisiones múltiples.
	\item \textbf{Dispersión por Encadenamiento}: Este método resuelve colisiones almacenando múltiples elementos colisionados en una misma posición de la tabla hash utilizando listas enlazadas u otras estructuras de datos, permitiendo el acceso eficiente a los elementos colisionados.
\end{enumerate}

La elección del método de dispersión depende de factores como la distribución de claves, el tamaño de la tabla hash y los requisitos de rendimiento del sistema operativo y las aplicaciones.

\subsection{Solución de Colisiones}

La solución de colisiones es un aspecto crítico de la dispersión en SOs y se refiere a las estrategias utilizadas para manejar situaciones donde múltiples elementos se asignan a la misma posición en una estructura de datos dispersa, como una tabla hash. Algunas estrategias comunes para la solución de colisiones son:

\begin{enumerate}
	\item \textbf{Separación por Encadenamiento}: Esta estrategia resuelve colisiones almacenando múltiples elementos colisionados en una misma posición de la tabla hash utilizando listas enlazadas, árboles u otras estructuras de datos. Cada posición de la tabla contiene una lista de elementos que han sido asignados a esa posición.
	\item \textbf{Dispersión Abierta}: También conocida como resolución de colisiones por reasignación, esta estrategia utiliza métodos como la dispersión cuadrática, la dispersión doble o la dispersión lineal para encontrar posiciones alternativas en la tabla hash cuando se producen colisiones.
	\item \textbf{Reasignación Dinámica}: En esta estrategia, se aumenta dinámicamente el tamaño de la estructura de datos dispersa (como la tabla hash) cuando se producen colisiones frecuentes, lo que reduce la probabilidad de colisiones y mejora el rendimiento del sistema operativo.
\end{enumerate}

\clearpage
\section{Quiz Complementario Actividad 13}

\begin{figure}[h]
	\centering
	\includegraphics[width=0.8\textwidth, height=0.7\textheight, keepaspectratio]{./img/quiz.jpg}
	\caption{Quiz: Algoritmos de Planificación de Disco}
\end{figure}

\clearpage
\section{Conclusión}

En resumidas palabras, un \textbf{sistema de archivos} es aquel sistema que utiliza un SO para establecer el funcionamiento y el uso de los archivos y directorios. Es muy importante debido a que permite al usuario interactuar de una manera muy cómoda y eficaz con la memoria principal, haciendo posible almacenar información y hacer uso de ella.

Para finalizar, si bien no es estrictamente necesario aprender la implementación de un sistema de archivos, es muy importante cómo funciona, cómo utilizarlo, entender cada una de las operaciones que admite dicho sistema de archivos para, de esta manera, ser capaz de trabajar de manera correcta con los archivos, aprovechando cada una de las funcionalidades que este nos provee.
% Referencias

\nocite{*} % Para incluir todas las referencias sin necesidad de citarlas

\clearpage
\bibliographystyle{apalike}
\bibliography{ref}

\end{document}
