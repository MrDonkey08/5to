\documentclass[12pt, a4paper]{article} % Formato de plantlla que vamos a utilizar

\usepackage[utf8]{inputenc}
\usepackage[spanish]{babel}
\usepackage{setspace}
\usepackage[margin=2.5cm, left=3cm, right=2cm, includefoot]{geometry}
\usepackage{graphicx} % Inserción de imágenes
\usepackage[dvipsnames, table, xcdraw]{xcolor}
\usepackage[most]{tcolorbox} % Inserción de cuadros en la portada
\usepackage{fancyhdr} % Definir el estilo de la página
\usepackage[hidelinks]{hyperref} % Gestión de hipervínculos
\usepackage{listings} % Para la inserción de código
\usepackage{parskip} % Arreglo de la tabulación en el documento
\usepackage[figurename=Fig.]{caption} % Cambiar el nombre del caption de las fotos
\usepackage{smartdiagram} % Inserción de Diagramas
\usepackage{zed-csp} % Inserción de esquemas
\usepackage{hyperref} % Para hipervínculos
\usepackage{setspace}
\usepackage{titlesec}
\usepackage{blindtext} % Solo para generar texto de relleno, puedes eliminar esta línea en tu documento final.
\usepackage{natbib}
\usepackage{array} % <-- Add this line for m{} column type


% Encabezado y pie de página
\pagestyle{fancy}
\fancyhf{}
\renewcommand{\headrulewidth}{0pt} % Elimina la línea del encabezado
%\fancyfoot[C,CO]{\thepage} % Número de página en el centro para páginas pares e impares
%\fancyhead[RO]{\textbf{\theTitle}} % Nombre de la materia en páginas impares
%\fancyhead[L]{\textbf{\theauthor}} % Título de la investigación en páginas pares
\setlength{\headheight}{15pt} % Ajuste necesario para evitar warnings

% Configuración de fuentes
\usepackage{times} % Times New Roman
%\usepackage{tgheros} % Tahoma
%\renewcommand{\familydefault}{\sfdefault} % Descomenta esta línea si usas Tahoma
%\usepackage{uarial} % Arial
%\renewcommand{\rmdefault}{phv} % Descomenta esta línea si usas Arial

% Ajustes de formato
\renewcommand{\baselinestretch}{1.15} % Espaciado de línea anterior
\setlength{\parskip}{6pt} % Espaciado de línea anterior
\setlength{\parindent}{0pt} % Sangría

% Declaración de colores
\definecolor{greenPortada}{HTML}{69A84F}
\definecolor{LightGray}{gray}{0.9}
\definecolor{codegreen}{rgb}{0, 0.6, 0}
\definecolor{codegray}{rgb}{0.5, 0.5, 0.5}
\definecolor{codepurple}{rgb}{0.58, 0, 0.82}
\definecolor{backcolour}{rgb}{0.95, 0.95, 0.92}

% Esilo de enlaces
\hypersetup{
	colorlinks=true,
	linkcolor=greenPortada,
	filecolor=greenPortada,
	urlcolor=greenPortada,
	pdftitle={Overleaf Example},
	pdfpagemode=FullScreen,
}

\urlstyle{same}

% Estilo de bloques de código
\lstdefinestyle{mystyle}{
	backgroundcolor=\color{backcolour},
	commentstyle=\color{codegreen},
	keywordstyle=\color{magenta},
	numberstyle=\tiny\color{codegray},
	stringstyle=\color{codepurple},
	basicstyle=\ttfamily\footnotesize,
	breakatwhitespace=false,
	breaklines=true,
	captionpos=b,
	keepspaces=true,
	numbers=left,
	numbersep=5pt,
	showspaces=false,
	showstringspaces=false,
	showtabs=false,
	tabsize=2
}

\lstset{style=mystyle}

% Declaración de variables
\newcommand{\logoUdg}{../../../../attachments/images/portada-udg.jpeg}
\newcommand{\logoCucei}{../../../../attachments/images/portada-cucei.jpeg}

% Datos de la Materia
\newcommand{\materia}{Sistemas Operativos}
\newcommand{\theTitle}{3. Diagrama y transiciones}
\newcommand{\profesor}{Violeta del Rocío Becerra Velázquez}
\newcommand{\theAuthor}{Alan Yahir Juárez Rubio}
\newcommand{\code}{218517809}
\newcommand{\carrera}{Ingeniería en Cómputo}
\newcommand{\seccion}{D04}
\newcommand{\startDate}{18 de febrero de 2024}

% Espaciado
\newcommand{\nl}{\par\vspace{0.4cm}}

\newcommand{\website}{https://hackthebox.eu} % Sitio Web

% Adicionales
\addto\captionsspanish{\renewcommand{\contentsname}{Índice}}
\setlength{\headheight}{60pt}
\pagestyle{fancy}

% Encabezado
\fancyhf{}
\lhead{
	\begin{minipage}[c][2cm][c]{1.3cm}
		\begin{flushleft}
			\includegraphics[width=5cm, height=1.4cm, keepaspectratio]{\logoUdg}
		\end{flushleft}
	\end{minipage}
	\begin{minipage}[c][2cm][c]{0.5\textwidth} % Adjust the height as needed
		\begin{flushleft}	
		{\ifodd\value{page}\materia\fi}
		\end{flushleft}
	\end{minipage}
}
\rhead{
		\begin{minipage}[c][2cm][c]{0.4\textwidth} % Adjust the height as needed
			\begin{flushright}
				{\ifodd\value{page}\else\theTitle\fi}
			\end{flushright}
		\end{minipage}
		\begin{minipage}[c][2cm][c]{1.3cm}
			\begin{flushright}
				\includegraphics[width=5cm, height=1.4cm, keepaspectratio]{\logoCucei}
			\end{flushright}
		\end{minipage}
}

% Pie de página
\rfoot{\ifodd\value{page}\theAuthor\fi}

\renewcommand{\headrulewidth}{3pt}
\renewcommand{\headrule}{\hbox to\headwidth{\color{greenPortada}\leaders\hrule height \headrulewidth\hfill}}

\renewcommand{\lstlistingname}{Código} % Para cambiar el caption de los código
%\renewcommand\thepage{\ifodd\value{page}\else\arabic{page}\fi}

\title{\theTitle}
\author{\theAuthor}

% Comienzo del Documento
\begin{document}
\cfoot{\ifodd\value{page}\else\thepage\fi} % Paginación
\setstretch{1}

\begin{titlepage}
	\centering
	{\huge\textbf{Universidad de Guadalajara}}\par\vspace{0.6cm}
	{\LARGE{Centro Universitario de Ciencias Exactas e Ingenierías}}\vfill
	
	\begin{figure}[h]
		\begin{minipage}[t]{0.45\textwidth}
			\centering
			\includegraphics[width=130px, height=300px, keepaspectratio]{\logoUdg}
		\end{minipage}
		\hfill
		\begin{minipage}[t]{0.45\textwidth}
			\centering
			\includegraphics[width=130px, height=300px, keepaspectratio]{\logoCucei}
		\end{minipage}
	\end{figure}\vfill
	
	{\Large{División de Tecnologías para la Integración CiberHumana}}\vfill
	{\Large\textbf{\materia}}\vfill
	\begin{figure}[h]
		\centering
		\begin{minipage}[t]{0.75\textwidth}
			{\Large
				\textbf{Profesor:} \profesor\nl
				\textbf{Alumno:} \theAuthor\nl
				\textbf{Código:} \code\nl
				\textbf{Carrera:} \carrera\nl
				\textbf{Sección:} \seccion
			}
		\end{minipage}
	\end{figure}\vfill
	{\LARGE{\textbf{\theTitle}}}\vfill
	
	\begin{tcolorbox}[colback=red!5!white, colframe=red!75!black]
		\centering
		Este documento contiene información sensible.\\
		No debería ser impreso o compartido con terceras entidades.
	\end{tcolorbox}\vfill
	{\large \startDate}\par
\end{titlepage}

% ïndices
\clearpage
\tableofcontents

\clearpage
\listoffigures
	
%\clearpage
\listoftables
% Fin de Índices

\clearpage

\section{Creación y Finalización de Procesos}

Un proceso no es más que una instancia de un programa en ejecución, incluyendo los valores actuales del contador del programa, registros y variables. A gran escala, existen dos tipos de procesos:

\begin{itemize}
	\item \textbf{\textit{Foreground}}: Aquellos procesos que, al ser ejecutados, son visibles para el usuario. El usuario interactúa directamente con ellos. P. ej: reproducción de video, uso de procesadores de texto (p. ej. word,), reproducir un videojuego, etc.
	\item \textbf{\textit{Background}}: Aquellos que, al ser ejecutados, no son visibles para el usuario. El usuario interactúa indirectamente o inclusive no interactúa con ellos. P. ej: mantenimiento del sistema (virus, limpieza, actualización...), descarga de archivos, \textit{deamons} (p. ej: servicios del sistema), etc.
\end{itemize}

\subsection{Creación de Procesos}

Si bien, en sistemas simples o de propósito específico, al ser iniciados, es posible que se encuentren presentes todos los servicios que podrían ser necesario, en sistemas de propósito general es esencial tener alguna forma de crear y finalizar procesos como sea necesario, durante la operación.

Existen cuatro eventos que causan la creación de procesos:

\begin{enumerate}
	\item \textbf{Inicialización del sistema}: Se crean procesos necesarios para el funcionamiento del sistema y procesos que han sido establecidos por programas y el usuario para ser creados al iniciar el sistema.
	\item \textbf{Proceso llamado por otro proceso:} El proceso en ejecución solicita la creación de otro.
	\item \textbf{Solicitud del usuario de crear un nuevo procesos}: El usuario ejecuta un programa, ya sea al abrir el ejecutable (doble click), al ejecutar un comando (programa) a través del CLI, etc.
	\item \textbf{Iniciación de un trabajo por lotes}: Programa o conjunto de programa que son enviados al SO para ser ejecutados sin necesidad de interacción del usuario.
\end{enumerate}

\subsection{Finalización de Procesos}

Después de que un proceso haya sido creado, comienza a realizar cualquiera que sea su trabajo hasta que este finaliza. Este puede terminar debido a una de las siguientes condiciones:

\begin{enumerate}
	\item \textbf{Salida normal (voluntario)}: El usuario decide salir/finalizar el programa,
	\item \textbf{Error de salida (voluntario)}: El proceso descubre un error fatal. P. ej. un usuario ejecuta un comando con un parámetro o atributo incorrecto. En programas con interfaz gráfica suelen lanzar una ventana con el error, en vez de finalizar el programa.
	\item \textbf{Error fatal (involuntario)}: Error causado por el procesos, a menudo por un \textit{bug} del programa. Algunos ejemplos son la ejecución de instrucciones ilegales, referencias de memorias inexistenes o división entre cero. En algunos sistemas (p. ej UNIX) un proceso puede decirle al SO que lo maneje como una interrupción para manejar ciertos errores por sí mismo.
	\item \textbf{Matado por otro proceso (involuntario)}: Un proceso solicita al sistema finalizar otro proceso. Los procesos que fueron creados por el proceso finalizado también finalizan.
\end{enumerate}

\section{Estados de un Proceso}

\begin{figure}[h]
	\centering
	\includegraphics[width=0.65\textwidth]{./img/diagram_of_process_state.jpg}
	\caption{Diagrama de 5 Estados}
\end{figure}

\begin{itemize}
	\item \textbf{Nuevo (\textit{New})}: Se crea un proceso nuevo
	\item \textbf{Listo (\textit{Ready})}: El proceso ha sido admitido y está en cola de espera
	\item \textbf{En ejecución (\textit{Running})}: El proceso está siendo ejecutado
	\item \textbf{En espera (\textit{Wating})}: El proceso ha sido suspendido debido a una espera de I/O
	\item \textbf{Finalizado (\textit{Terminated})}: El proceso ha finalizado
\end{itemize}

\section{Algoritmos de Planificación}

En los sistemas de multiprogramaicón, la \textbf{planificación} del procesador se encarga de seleccionar entre todos los procesos en espera aquel que se le asignará el procesador.

Fundamentalmente, la planificación es una cuestión de manejo de colas para minimizar la demora de los procesos en éstas y optimizar el rendimiento.

\subsection{Políticas de Planificación}

\begin{itemize}
	\item \textbf{Apropiativas}: El proceso en ejecución puede ser interrumpido y pasar a estado listo. Esta decisión la puede tomar el sistema operativo cuando llega un nuevo proceso, cuando un proceso bloqueado pasa a estado listo, cuando se activa un proceso suspendido, o periódicamente al producirse una interrupción del reloj. 
	\item \textbf{No Apropiativas}: Cuando un proceso empieza a ejecutarse, solo se le puede retirar la CPU si se bloquea o termina.
\end{itemize}
\begin{table}[h]
	\centering
	\renewcommand{\arraystretch}{1.2} % <-- Adjust vertical spacing
	\begin{tabular}{|m{1.8cm}|m{6cm}|m{2cm}|m{2cm}|m{2cm}|m{2cm}}\hline
		\textbf{Algoritmo} & \textbf{Descripción} & \textbf{Política} & \textbf{Ventaja} & \textbf{Desventaja}\\ \hline
		RR: Round Robin & Los procesos se ejecutan de manera secuencial en turnos, y si un proceso no se completa dentro de su quantum, se suspende y se coloca al final de la cola listos para ser ejecutado nuevamente & No Apropiativo & Evita la inanición de procesos largos & Tiempo de respuesta más largo\\ \hline
		FCFS: First-Come, First-Served & El proceso que llega primero se ejecuta primero, y los demás esperan en la cola hasta que su turno llegue. &No Apropiativo & Simple de implementar & Puede causar inanición de procesos\\ \hline
		SRT: Shortest Remaining Time & SRT verifica continuamente si hay un proceso más corto en la cola listos y lo ejecuta si es así & Apropiativo & Minimiza el tiempo de respuesta & Puede causar inanición de procesos largos\\ \hline
		SJF: Shortest Job First & Se selecciona para ejecución el proceso con el tiempo de ejecución más corto en la cola listos & No Apropiativo & Minimiza el tiempo de espera & No es justo, procesos largos pueden esperar\\ \hline
		Prioridades & Cada proceso tiene una prioridad asignada, y el algoritmo selecciona para ejecución el proceso con la prioridad más alta en la cola listos & Apropiativo / No Apropiativo & Permite dar prioridad a procesos importantes & Puede causar inanición de procesos de baja prioridad\\ \hline
		Colas Múltiples & Los procesos nuevos ingresan a la cola de prioridad más alta y, si no se completan dentro de un cierto límite de tiempo, se mueven a una cola de prioridad más baja & Apropiativo & Permite asignar diferentes niveles de prioridad & Mayor complejidad de implementación\\ \hline
		RR mejorado & Ajusta dinámicamente el quantum de tiempo según las necesidades de los procesos & Apropiativo & Ajusta el quantum según las necesidades & Puede tener un mayor overhead de procesamiento\\ \hline
		HRRN: Highest Response Ratio Next & Se priorizan los procesos en función de su relación entre el tiempo de respuesta esperado y el tiempo de ejecución restante. El proceso con la relación más alta se selecciona para ejecución primero & No Apropiativo & Procesos con alta relación respuesta/tiempo & Puede provocar inanición de procesos largos\\ \hline
	\end{tabular}
	\caption{Descripción de algoritmos de planificación de CPU}
\end{table}

\clearpage
\section*{Cuestionario}

\begin{enumerate}
	\item Describa en qué consisten los algoritmos de planificación No Apropiativos.
		
		Son aquellos que, una vez un proceso ha comenzado a ejecutarse, no habrá interrupciones, es decir, no se detendrá o hasta que haya finalizado o hasta que se bloquee.

	\item Explique cómo obtener cada uno de los tiempos solicitados en el punto 9 del programa 3 (actividad de aprendizaje 6).

	\begin{enumerate}
		\item \textbf{Llegada}: Al ejecutar el proceso obtengo la hora del sistema.
		\item \textbf{Finalización}: Una vez finalizado obtengo la hora del sistema.
		\item \textbf{Retorno}: Calculo la diferencia entre la hora de llegada y de finalización del proceso.
		\item \textbf{Respuesta}: Registro la hora en la que llega, la de cuando es atendido y cálculo la diferencia.
		\item \textbf{Espera}: Registro la hora en la que llega, la de cuando es atendido y cálculo la diferencia.
		\item \textbf{Servicio}: Si no hubo interrupciones, el TME, sino, registro cada una de las horas en las que ingresa al procesador, cálculos sus respectivas diferencias con sus respectivas horas de suspensión/finalización y las sumo.
	\end{enumerate}

	\item ¿Qué significa BCP? 

		BCP significa Bloque de Control de Procesos (BCP). Es una estructura de datos en un sistema operativo que contiene información sobre un proceso en ejecución o en espera.
\end{enumerate}

\clearpage
\section*{Conclusión}

En retrospectiva, la gestión de procesos dentro de un SO debe estar bien estructurada e implementada para poder tener un correcto funcionamiento y un buen aprovechamiento de los recursos.

Para finalizar, desde mi punto de vista es de vital importancia entender el cómo se crean, finalizan los procesos y, además, entender cada una de las etapas o estados de un proceso.


% Referencias

\nocite{*} % Para incluir todas las referencias sin necesidad de citarlas

\clearpage
\bibliographystyle{apalike}
\bibliography{ref}

\end{document}
