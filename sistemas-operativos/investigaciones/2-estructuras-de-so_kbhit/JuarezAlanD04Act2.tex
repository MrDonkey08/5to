\documentclass[12pt, a4paper]{article} % Formato de plantlla que vamos a utilizar

\usepackage[utf8]{inputenc}
\usepackage[spanish]{babel}
\usepackage{setspace}
\usepackage[margin=2.5cm, left=3cm, right=2cm, includefoot]{geometry}
\usepackage{graphicx} % Inserción de imágenes
\usepackage[dvipsnames, table, xcdraw]{xcolor}
\usepackage[most]{tcolorbox} % Inserción de cuadros en la portada
\usepackage{fancyhdr} % Definir el estilo de la página
\usepackage[hidelinks]{hyperref} % Gestión de hipervínculos
\usepackage{listings} % Para la inserción de código
\usepackage{parskip} % Arreglo de la tabulación en el documento
\usepackage[figurename=Fig.]{caption} % Cambiar el nombre del caption de las fotos
\usepackage{smartdiagram} % Inserción de Diagramas
\usepackage{zed-csp} % Inserción de esquemas
\usepackage{hyperref} % Para hipervínculos
\usepackage{setspace}
\usepackage{titlesec}
\usepackage{blindtext} % Solo para generar texto de relleno, puedes eliminar esta línea en tu documento final.
\usepackage{natbib}

% Encabezado y pie de página
\pagestyle{fancy}
\fancyhf{}
\renewcommand{\headrulewidth}{0pt} % Elimina la línea del encabezado
%\fancyfoot[C,CO]{\thepage} % Número de página en el centro para páginas pares e impares
%\fancyhead[RO]{\textbf{\theTitle}} % Nombre de la materia en páginas impares
%\fancyhead[L]{\textbf{\theauthor}} % Título de la investigación en páginas pares
\setlength{\headheight}{15pt} % Ajuste necesario para evitar warnings

% Configuración de fuentes
\usepackage{times} % Times New Roman
%\usepackage{tgheros} % Tahoma
%\renewcommand{\familydefault}{\sfdefault} % Descomenta esta línea si usas Tahoma
%\usepackage{uarial} % Arial
%\renewcommand{\rmdefault}{phv} % Descomenta esta línea si usas Arial

% Ajustes de formato
\renewcommand{\baselinestretch}{1.15} % Espaciado de línea anterior
\setlength{\parskip}{6pt} % Espaciado de línea anterior
\setlength{\parindent}{0pt} % Sangría

% Declaración de colores
\definecolor{greenPortada}{HTML}{69A84F}
\definecolor{LightGray}{gray}{0.9}
\definecolor{codegreen}{rgb}{0, 0.6, 0}
\definecolor{codegray}{rgb}{0.5, 0.5, 0.5}
\definecolor{codepurple}{rgb}{0.58, 0, 0.82}
\definecolor{backcolour}{rgb}{0.95, 0.95, 0.92}

% Esilo de enlaces
\hypersetup{
	colorlinks=true,
	linkcolor=greenPortada,
	filecolor=greenPortada,
	urlcolor=greenPortada,
	pdftitle={Overleaf Example},
	pdfpagemode=FullScreen,
}

\urlstyle{same}

% Estilo de bloques de código
\lstdefinestyle{mystyle}{
	backgroundcolor=\color{backcolour},
	commentstyle=\color{codegreen},
	keywordstyle=\color{magenta},
	numberstyle=\tiny\color{codegray},
	stringstyle=\color{codepurple},
	basicstyle=\ttfamily\footnotesize,
	breakatwhitespace=false,
	breaklines=true,
	captionpos=b,
	keepspaces=true,
	numbers=left,
	numbersep=5pt,
	showspaces=false,
	showstringspaces=false,
	showtabs=false,
	tabsize=2
}

\lstset{style=mystyle}

% Declaración de variables
\newcommand{\logoUdg}{../../../../attachments/images/portada-udg.jpeg}
\newcommand{\logoCucei}{../../../../attachments/images/portada-cucei.jpeg}

% Datos de la Materia
\newcommand{\materia}{Sistemas Operativos}
\newcommand{\theTitle}{2. Estructuras de so, kbhit}
\newcommand{\profesor}{Violeta del Rocío Becerra Velázquez}
\newcommand{\theAuthor}{Alan Yahir Juárez Rubio}
\newcommand{\code}{218517809}
\newcommand{\carrera}{Ingeniería en Cómputo}
\newcommand{\seccion}{D04}
\newcommand{\startDate}{04 de febrero de 2024}

% Espaciado
\newcommand{\nl}{\par\vspace{0.4cm}}

\newcommand{\website}{https://hackthebox.eu} % Sitio Web

% Adicionales
\addto\captionsspanish{\renewcommand{\contentsname}{Índice}}
\setlength{\headheight}{60pt}
\pagestyle{fancy}

% Encabezado
\fancyhf{}
\lhead{
	\begin{minipage}[c][2cm][c]{1.3cm}
		\begin{flushleft}
			\includegraphics[width=5cm, height=1.4cm, keepaspectratio]{\logoUdg}
		\end{flushleft}
	\end{minipage}
	\begin{minipage}[c][2cm][c]{0.5\textwidth} % Adjust the height as needed
		\begin{flushleft}	
		{\ifodd\value{page}\materia\fi}
		\end{flushleft}
	\end{minipage}
}
\rhead{
		\begin{minipage}[c][2cm][c]{0.4\textwidth} % Adjust the height as needed
			\begin{flushright}
				{\ifodd\value{page}\else\theTitle\fi}
			\end{flushright}
		\end{minipage}
		\begin{minipage}[c][2cm][c]{1.3cm}
			\begin{flushright}
				\includegraphics[width=5cm, height=1.4cm, keepaspectratio]{\logoCucei}
			\end{flushright}
		\end{minipage}
}

% Pie de página
\rfoot{\ifodd\value{page}\theAuthor\fi}

\renewcommand{\headrulewidth}{3pt}
\renewcommand{\headrule}{\hbox to\headwidth{\color{greenPortada}\leaders\hrule height \headrulewidth\hfill}}

\renewcommand{\lstlistingname}{Código} % Para cambiar el caption de los código
%\renewcommand\thepage{\ifodd\value{page}\else\arabic{page}\fi}

\title{\theTitle}
\author{\theAuthor}

% Comienzo del Documento
\begin{document}
\cfoot{\ifodd\value{page}\else\thepage\fi} % Paginación
\setstretch{1}

\begin{titlepage}
	\centering
	{\huge\textbf{Universidad de Guadalajara}}\par\vspace{0.6cm}
	{\LARGE{Centro Universitario de Ciencias Exactas e Ingenierías}}\vfill
	
	\begin{figure}[h]
		\begin{minipage}[t]{0.45\textwidth}
			\centering
			\includegraphics[width=130px, height=300px, keepaspectratio]{\logoUdg}
		\end{minipage}
		\hfill
		\begin{minipage}[t]{0.45\textwidth}
			\centering
			\includegraphics[width=130px, height=300px, keepaspectratio]{\logoCucei}
		\end{minipage}
	\end{figure}\vfill
	
	{\Large{División de Tecnologías para la Integración CiberHumana}}\vfill
	{\Large\textbf{\materia}}\vfill
	\begin{figure}[h]
		\centering
		\begin{minipage}[t]{0.75\textwidth}
			{\Large
				\textbf{Profesor:} \profesor\nl
				\textbf{Alumno:} \theAuthor\nl
				\textbf{Código:} \code\nl
				\textbf{Carrera:} \carrera\nl
				\textbf{Sección:} \seccion
			}
		\end{minipage}
	\end{figure}\vfill
	{\LARGE{\textbf{\theTitle}}}\vfill
	
	\begin{tcolorbox}[colback=red!5!white, colframe=red!75!black]
		\centering
		Este documento contiene información sensible.\\
		No debería ser impreso o compartido con terceras entidades.
	\end{tcolorbox}\vfill
	{\large \startDate}\par
\end{titlepage}

% ïndices
\clearpage
\tableofcontents

%\clearpage
%\listoffigures
	
%\clearpage
%\listoftables
% Fin de Índices

\clearpage
\section{Modelos de Sistemas Operativos}

\subsection{Monolítico}

Es por mucho la organización más común. Consiste en que el SO corra como un solo programa en \textit{Kernel Mode}. El SO es escrito como una colección de procedimientos, enlazados entre sí en un único programa binario ejecutable grande. Cuando este modelo es utilizdo, cada proceso en el sistema es libre de llamar a otro. Si bien esto puede llegar a ser eficiente, el hecho de tener miles de procesos que pueden ser llamados sin restricciones hace que sea un sistema desagradable y difícil de entender. Además, un crasheo en uno de los procesos tumbaría por completo el SO.

No obstante, en Sistemas Monolíticos es posible tener algo de estructura. Los servicios (llamadas del sistema) proporcionados por el SO son solicitados al poner los parámetros en un lugar bien definidos (p. ej., en el \textit{stack}) y luego ejecutando un \textit{trap instruction}. Esta instrucción cambia la máquina de \textit{user mode} a \textit{kernel mode} y transfiere control al SO. Después, el SO busca (\textit{fetches}) los parámetros y determina cuál llamada del Sistema se llevará a cabo.

Esta organización sugiere una estructura básica para el SO:

\begin{enumerate}
	\item Un Programa principal que invoca el procedimiento del sistema solicitado.
	\item Un conjunto de procedimientos de servicio que llevan a cabo llamadas del sistema.
	\item Un conjunto de procedmientos de utilidad que ayuda a los procedimientos de servicios.
\end{enumerate}

\subsection{Cliente-Servidor}

Una pequeña variación de la idea del microkernel es la distinción de dos clases de procesos:

\begin{itemize}
	\item \textbf{Servidores}: Aquellos que proveen algún servicio.
	\item \textbf{Clientes}: Aquellos que utilizan los anteriores.
\end{itemize}

La comunicación entre clientes y servidores es a menudo mediante el paso de mensajes. Para obtener un servisio, un proceso de cliente construye un mensaje diciendo qué es lo que requiere y se lo envía al proceso de servicio apropiado. El servicio hace su trabajo y envía devuelta una respuesta.

Una obvia generalización de esta idea es tener los clientes y servidores en diferentes computadoras, conectadas por una red local o de área extensa. Como los clientes se comunican con servidorse por medio del envío de mensajes, los clientes no necesistan saber si lo s mensajes se gestionan localmente en sus propias máquinas o si se envían a través de una red de servidores en una máquina remota. En lo que respecta al cliente,

\subsection{Máquina Virtual}

Anteriormente, cuando en una máquina se requerirían varios SO, uno estaba forzado a que estos fuesen independientes que solo se pudiera correr uno a la vez. Una estrategía efectiva y novedosa para lidiar con este problema es la \textbf{virtualización}. La virtualización permite a la máquina correr, desde un SO host, diferentes SO simultáneamente.

Por otra parte, una \textbf{máquina virtual} es una conjunto de dispositivos de hardware emulados en el cual corre el SO. Para el SO le es indistinguible una máquina física de una virtual, lo cual permite que este funcione de la misma manera.

Cabe mencionar que, para crear, gestionar y utilizar máquinas virtuales, es necesario tener un gestor de máquinas virtuales.

\subsection{Capas}

Una forma de estructurar el sistema operativo es mediante la \textbf{estratificación}, que consiste en romper el sistema en estratos, niveles o capas, cada uno de ellos construido sobre los anteriores. Una capa de un sistema operativo es una implementación de un objeto abstracto que permite la encapsulación de datos y las operaciones que pueden manipular esos datos. En un sistema de este tipo, la capa más baja (capa 0) es el hardware, y la más alta (capa N) es la interfaz con el usuario.

\subsection{Híbrido}

Un SO híbrido es aquel que combina características de diferentes tipos de SO's, como SO's de tiempo compartido (como Unix/Linux) y SO's de tiempo real (como QNX). Estos SO's suelen ser diseñados para funcionar en una amplia gama de dispositivos y entornos, y pueden manejar tanto aplicaciones de tiempo real como aplicaciones de propósito general.

Los SO's híbridos suelen ofrecer un equilibrio entre la capacidad de respuesta en tiempo real y la capacidad de ejecutar múltiples tareas de manera eficiente. Pueden manejar eventos críticos en tiempo real, como el control de dispositivos en sistemas embebidos, al tiempo que admiten la ejecución de aplicaciones de propósito general.

Una implementación común de un sistema operativo híbrido es el uso de un kernel (núcleo) que combina características de kernels monolíticos y kernels microkernel. Por ejemplo, el kernel puede manejar las operaciones esenciales del sistema operativo de manera eficiente, mientras que ciertas funcionalidades, como los controladores de dispositivos, se ejecutan en modo de usuario para mejorar la estabilidad y la seguridad del sistema.

\clearpage
\section{Arch Linux}

Arch Linux. Es una distribución de Linux independiente, de propósito general, desarrollada para las arquitecturas i686 y x86-64 diseñada para ser liviana y simple. El enfoque de su diseño se centra en simplicidad, elegancia, coherencia de código y minimalismo. Arch Linux utiliza su propio gestor de paquetes, Pacman que le permite a los usuarios administrar y personalizar desde software oficial hasta paquetes personales. Arch usa un modelo "rolling release" el cual permite la instalación del sistema una sola vez y actualizaciones perpetuas, de modo que el usuario no tiene que reinstalar el sistema de una versión a la siguiente.

\subsection{Historia}

Judd Vinet comenzó el desarrollo de Arch Linux a principio del año 2001 y su primera versión fue publicada el 11 de marzo de 2002. Arch Linux fue inspirado en la elegancia y simplicidad de Slackware y CRUX. Además de la creación de la distribución, Judd también creó un programa que permitía instalar, remover y actualizar paquetes.

A finales del año 2007, Judd Vinet se retiró de la participación activa como desarrollador y cedió el proyecto a Aaron Griffin el cual continúa como líder en la actualidad.

\subsection{Servicios que presta}

Ofrece acceso a repositorios de software extensos y actualizados, herramientas de gestión de paquetes como Pacman y AUR (Arch User Repository), y una comunidad activa que proporciona soporte y documentación detallada.

\subsection{Objetivos}

Los principales objetivos de Arch Linux son la simplicidad, la elegancia del diseño, la minimalidad y la limpieza del código. Busca mantener un sistema base liviano y flexible, permitiendo a los usuarios construir su entorno de trabajo personalizado según sus necesidades y preferencias.

\subsection{Funciones}

Arch Linux se destaca por su enfoque rolling release, lo que significa que se actualiza continuamente sin necesidad de reinstalaciones completas entre versiones. Esto garantiza que los usuarios siempre tengan acceso a las últimas versiones de software. Además, Arch permite una instalación modular, lo que significa que los usuarios pueden instalar solo los paquetes y componentes que necesiten, evitando la sobrecarga de software no deseado.

\subsection{Estructura}

La estructura de Arch Linux se basa en un sistema de inicio minimalista (init), lo que permite a los usuarios tener un mayor control sobre los procesos y servicios que se ejecutan en su sistema. La distribución se organiza en torno a un sistema de archivos simple y coherente, con directorios bien definidos para diferentes tipos de archivos y configuraciones. Además, Arch utiliza systemd como sistema de inicio y gestión de servicios, lo que ofrece una mayor velocidad y eficiencia en comparación con los sistemas init tradicionales.

\clearpage
\section{Cuestionario}

\begin{enumerate}
	\item ¿Qué significa JCL?

		Significa Job Control Language o, en español, Lenguaje de Control de Trabajo. Es un lenguaje que permite codificar las instrucciones necesarias para la ejecución de un proceso en batch. Estas instrucciones son interpretadas y ejecutadsas por Gestor de Trabajos (JES).
		
	\item Escriba la diferencia entre el procesamiento por lotes y el de procesamiento por lotes con multiprogramación.

		Mientras que el procesamiento por lotes simple consiste en ejecutar de manera secuencial, el procecamiento por lotes con multiprogramación, es capaz de suspender una instrucción que espera una E/S para proseguir con otras instrucciones, esto con el fin de no desperdiciar el tiempo de esperar. En resumén, el simple ejecuta instrucciones secuencialmente, mientras que multiprogramación superpone actividades para un mayor aprovechamiento de los recursos.

	\item Escribe una de las utilidades de la interrupción int86 en C.

		Realizar llamadas al BIOS (Sistema Básico de Entrada/Salida) de la computadora. Por ejemplo, la interrupción int86 puede usarse para interactuar con hardware específico, como la pantalla o el teclado, accediendo a funciones de bajo nivel proporcionadas por el BIOS.

	\item ¿Para qué sirve la función Kbhit?

		Sirve para comprobar si se ha presionado una tecla en el teclado sin bloquear la ejecución del programa. Retorna un valor distinto de cero si hay una tecla disponible para ser leída mediante la función getch() o getchar(), de lo contrario, retorna cero.

	\item Investigue el equivalente de Kbhit (utilizado en c) en otros dos lenguajes de programación y escríbalos.

		\begin{enumerate}
			\item En Java, puedes usar la clase java.awt.event.KeyEvent junto con java.awt.event.KeyListener para detectar eventos de teclado. Sin embargo, Java no tiene una función directa equivalente a Kbhit en C, ya que se basa en un modelo de manejo de eventos más orientado a objetos.

			\item En Python, la biblioteca msvcrt proporciona una función similar llamada kbhit(), que devuelve True si se ha presionado una tecla, de lo contrario, devuelve False.
		\end{enumerate}
\end{enumerate}

% Referencias

\nocite{*} % Para incluir todas las referencias sin necesidad de citarlas

\clearpage
\bibliographystyle{apalike}
\bibliography{ref}

\end{document}
