\documentclass[12pt, a4paper]{article} % Formato de plantlla que vamos a utilizar

\usepackage[utf8]{inputenc}
\usepackage[spanish]{babel}
\usepackage{setspace}
\usepackage[margin=2.5cm, left=3cm, right=2cm, includefoot]{geometry}
\usepackage{graphicx} % Inserción de imágenes
\usepackage[dvipsnames, table, xcdraw]{xcolor}
\usepackage[most]{tcolorbox} % Inserción de cuadros en la portada
\usepackage{fancyhdr} % Definir el estilo de la página
\usepackage[hidelinks]{hyperref} % Gestión de hipervínculos
\usepackage{listings} % Para la inserción de código
\usepackage{parskip} % Arreglo de la tabulación en el documento
\usepackage[figurename=Fig.]{caption} % Cambiar el nombre del caption de las fotos
\usepackage{smartdiagram} % Inserción de Diagramas
\usepackage{zed-csp} % Inserción de esquemas
\usepackage{hyperref} % Para hipervínculos
\usepackage{setspace}
\usepackage{titlesec}
\usepackage{blindtext} % Solo para generar texto de relleno, puedes eliminar esta línea en tu documento final.
\usepackage{natbib}
\usepackage{array} % <-- Add this line for m{} column type

% Encabezado y pie de página
\pagestyle{fancy}
\fancyhf{}
\renewcommand{\headrulewidth}{0pt} % Elimina la línea del encabezado
\setlength{\headheight}{15pt} % Ajuste necesario para evitar warnings

% Configuración de fuentes
\usepackage{times} % Times New Roman

% Ajustes de formato
\renewcommand{\baselinestretch}{1.15} % Espaciado de línea anterior
\setlength{\parskip}{6pt} % Espaciado de línea anterior
\setlength{\parindent}{0pt} % Sangría

% Declaración de colores
\definecolor{greenPortada}{HTML}{69A84F}
\definecolor{LightGray}{gray}{0.9}
\definecolor{codegreen}{rgb}{0, 0.6, 0}
\definecolor{codegray}{rgb}{0.5, 0.5, 0.5}
\definecolor{codepurple}{rgb}{0.58, 0, 0.82}
\definecolor{backcolour}{rgb}{0.95, 0.95, 0.92}

% Esilo de enlaces
\hypersetup{
	colorlinks=true,
	linkcolor=greenPortada,
	filecolor=greenPortada,
	urlcolor=greenPortada,
	pdftitle={Overleaf Example},
	pdfpagemode=FullScreen,
}

\urlstyle{same}

% Estilo de bloques de código
\lstdefinestyle{mystyle}{
	backgroundcolor=\color{backcolour},
	commentstyle=\color{codegreen},
	keywordstyle=\color{magenta},
	numberstyle=\tiny\color{codegray},
	stringstyle=\color{codepurple},
	basicstyle=\ttfamily\footnotesize,
	breakatwhitespace=false,
	breaklines=true,
	captionpos=b,
	keepspaces=true,
	numbers=left,
	numbersep=5pt,
	showspaces=false,
	showstringspaces=false,
	showtabs=false,
	tabsize=2
}

\lstset{style=mystyle}

% Declaración de variables
\newcommand{\logoUdg}{../../../../attachments/images/portada-udg.jpeg}
\newcommand{\logoCucei}{../../../../attachments/images/portada-cucei.jpeg}

% Datos de la Materia
\newcommand{\materia}{Sistemas Operativos}
\newcommand{\theTitle}{6. Técnicas de Administación de Memoria}
\newcommand{\profesor}{Violeta del Rocío Becerra Velázquez}
\newcommand{\theAuthor}{Alan Yahir Juárez Rubio}
\newcommand{\code}{218517809}
\newcommand{\carrera}{Ingeniería en Cómputo}
\newcommand{\seccion}{D04}
\newcommand{\startDate}{14 de abril de 2024}

% Espaciado
\newcommand{\nl}{\par\vspace{0.4cm}}

% Adicionales
\addto\captionsspanish{\renewcommand{\contentsname}{Índice}}
\setlength{\headheight}{60pt}
\pagestyle{fancy}

% Encabezado
\fancyhf{}
\lhead{
	\begin{minipage}[c][2cm][c]{1.3cm}
		\begin{flushleft}
			\includegraphics[width=5cm, height=1.4cm, keepaspectratio]{\logoUdg}
		\end{flushleft}
	\end{minipage}
	\begin{minipage}[c][2cm][c]{0.5\textwidth} % Adjust the height as needed
		\begin{flushleft}	
		{\ifodd\value{page}\materia\fi}
		\end{flushleft}
	\end{minipage}
}
\rhead{
		\begin{minipage}[c][2cm][c]{0.4\textwidth} % Adjust the height as needed
			\begin{flushright}
				{\ifodd\value{page}\else\theTitle\fi}
			\end{flushright}
		\end{minipage}
		\begin{minipage}[c][2cm][c]{1.3cm}
			\begin{flushright}
				\includegraphics[width=5cm, height=1.4cm, keepaspectratio]{\logoCucei}
			\end{flushright}
		\end{minipage}
}

% Pie de página
\rfoot{\ifodd\value{page}\theAuthor\fi}

\renewcommand{\headrulewidth}{3pt}
\renewcommand{\headrule}{\hbox to\headwidth{\color{greenPortada}\leaders\hrule height \headrulewidth\hfill}}

\renewcommand{\lstlistingname}{Código} % Para cambiar el caption de los código
%\renewcommand\thepage{\ifodd\value{page}\else\arabic{page}\fi}

\title{\theTitle}
\author{\theAuthor}

% Comienzo del Documento
\begin{document}
\cfoot{\ifodd\value{page}\else\thepage\fi} % Paginación
\setstretch{1}

\begin{titlepage}
	\centering
	{\huge\textbf{Universidad de Guadalajara}}\par\vspace{0.6cm}
	{\LARGE{Centro Universitario de Ciencias Exactas e Ingenierías}}\vfill
	
	\begin{figure}[h]
		\begin{minipage}[t]{0.45\textwidth}
			\centering
			\includegraphics[width=130px, height=300px, keepaspectratio]{\logoUdg}
		\end{minipage}
		\hfill
		\begin{minipage}[t]{0.45\textwidth}
			\centering
			\includegraphics[width=130px, height=300px, keepaspectratio]{\logoCucei}
		\end{minipage}
	\end{figure}\vfill
	
	{\Large{División de Tecnologías para la Integración CiberHumana}}\vfill
	{\Large\textbf{\materia}}\vfill
	\begin{figure}[h]
		\centering
		\begin{minipage}[t]{0.75\textwidth}
			{\Large
				\textbf{Profesor:} \profesor\nl
				\textbf{Alumno:} \theAuthor\nl
				\textbf{Código:} \code\nl
				\textbf{Carrera:} \carrera\nl
				\textbf{Sección:} \seccion
			}
		\end{minipage}
	\end{figure}\vfill
	{\LARGE{\textbf{\theTitle}}}\vfill
	
	\begin{tcolorbox}[colback=red!5!white, colframe=red!75!black]
		\centering
		Este documento contiene información sensible.\\
		No debería ser impreso o compartido con terceras entidades.
	\end{tcolorbox}\vfill
	{\large \startDate}\par
\end{titlepage}

% ïndices
\clearpage
\tableofcontents

\clearpage
\listoffigures
	
%\clearpage
%\listoftables
% Fin de Índices

\clearpage

\begin{figure}[h]
	\centering
	\includegraphics[width=0.8\textwidth, height=0.9\textheight, keepaspectratio]{./img/1-manejo-de-memoria.jpg}
	\caption{Técnicas para el Manejo de Memoria}
\end{figure}

\clearpage
\section{Cuestionario}

\begin{enumerate}
	\item ¿En que consiste la paginación simple?

		La paginación simple es una técnica de administración de memoria que divide la memoria física y virtual en páginas de tamaño fijo. Cuando un programa necesita memoria, se asignan páginas contiguas en la memoria física para almacenar sus datos y código.

	\item ¿En qué consiste la Técnica de Particiones Estáticas?

		La técnica de Particiones Estáticas divide la memoria en bloques de tamaño fijo al iniciar el sistema. Cada bloque se asigna a un proceso y no se puede modificar el tamaño de las particiones durante la ejecución.

	\item ¿En qué consiste la Técnica de Particiones Dinámicas?

		La Técnica de Particiones Dinámicas asigna memoria de manera flexible a los procesos según sus necesidades. Las particiones se crean y se eliminan dinámicamente durante la ejecución, lo que permite un uso más eficiente de la memoria.

	\item Escriba en que consiste la Memoria Virtual

		La memoria virtual consiste en un espacio reservado (partición o archivo de intercabio) del almacenamiento secundario (HDD, SSD) para que, una vez que el almacenamiento principal no sea suficiente, utilice este espacio como una expansión, permitiendo guardar y acceder a archivos como si fuese almacenamiento principal. La desventaja de esta técnica es que el almacenamiento secundario es más lento que el principal.

	\item Describa el funcionamiento de paginación con memoria virtual.

		En la paginación con memoria virtual, la memoria se divide en páginas tanto en la memoria física como en la virtual. Cuando se necesita una página que no está en la RAM, el sistema operativo la mueve desde el almacenamiento secundario a la RAM, utilizando técnicas como el swapping.

	\item Describa el funcionamiento de Segmentación con memoria virtual.

		En la segmentación con memoria virtual, los programas se dividen en segmentos lógicos como el código, los datos y la pila. Cada segmento se maneja de manera independiente, permitiendo que se asignen de forma dinámica a la memoria física o virtual según sea necesario.

	\item ¿Cuáles son los elementos que conforman la tabla de páginas?

	\begin{itemize}
		\item Número de página.
		\item Número de marco de página (dirección física).
		\item Bits de control para gestionar la validez de la página, los permisos de acceso, entre otros.
	\end{itemize}

	\item ¿Que son los buffers, cuál es su importancia y su manejo?

		Los buffers son áreas de memoria utilizadas para almacenar datos temporales durante las operaciones de entrada/salida (E/S). Son importantes porque ayudan a mejorar la eficiencia al evitar que los procesos se bloqueen mientras esperan la finalización de operaciones E/S lentas. El manejo de buffers incluye su asignación, liberación y control para garantizar un flujo eficiente de datos en el sistema.
\end{enumerate}

\clearpage
\section{Conclusión}

En retrospectiva, la gestión de la memoria es una parte fundamental del sistema, debido a que permite que la memoria sea bien distribuida por cada uno de los procesos. Si bien existen diversas técnicas para gestionar la memoria, ninguna es perfecta, cada una de ellas tiene un enfoque y un uso diferente

Para finalizar, me es relevante mencionar que, si bien no es del todo necesario entender exhaustivamente el funcionamiento completo de cada uno de estos métodos, es importante comprender la importancia de cada uno y el cómo es que el SO se encarga de administrar los recursos, la memoria.
% Referencias

\nocite{*} % Para incluir todas las referencias sin necesidad de citarlas

\clearpage
\bibliographystyle{apalike}
\bibliography{ref}

\end{document}
