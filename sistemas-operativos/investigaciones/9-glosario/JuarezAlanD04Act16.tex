\documentclass[12pt, a4paper]{article} % Formato de plantlla que vamos a utilizar

\usepackage[utf8]{inputenc}
\usepackage[spanish]{babel}
\usepackage{setspace}
\usepackage[margin=2.5cm, left=3cm, right=2cm, includefoot]{geometry}
\usepackage{graphicx} % Inserción de imágenes
\usepackage[dvipsnames, table, xcdraw]{xcolor}
\usepackage[most]{tcolorbox} % Inserción de cuadros en la portada
\usepackage{fancyhdr} % Definir el estilo de la página
\usepackage[hidelinks]{hyperref} % Gestión de hipervínculos
\usepackage{listings} % Para la inserción de código
\usepackage{parskip} % Arreglo de la tabulación en el documento
\usepackage[figurename=Fig.]{caption} % Cambiar el nombre del caption de las fotos
\usepackage{smartdiagram} % Inserción de Diagramas
\usepackage{zed-csp} % Inserción de esquemas
\usepackage{hyperref} % Para hipervínculos
\usepackage{setspace}
\usepackage{titlesec}
\usepackage{blindtext} % Solo para generar texto de relleno, puedes eliminar esta línea en tu documento final.
\usepackage{natbib}
\usepackage{array} % <-- Add this line for m{} column type
\usepackage{enumitem}

% Encabezado y pie de página
\pagestyle{fancy}
\fancyhf{}
\renewcommand{\headrulewidth}{0pt} % Elimina la línea del encabezado
\setlength{\headheight}{15pt} % Ajuste necesario para evitar warnings

% Configuración de fuentes
\usepackage{times} % Times New Roman

% Ajustes de formato
\renewcommand{\baselinestretch}{1.15} % Espaciado de línea anterior
\setlength{\parskip}{6pt} % Espaciado de línea anterior
\setlength{\parindent}{0pt} % Sangría

% Declaración de colores
\definecolor{greenPortada}{HTML}{69A84F}
\definecolor{LightGray}{gray}{0.9}
\definecolor{codegreen}{rgb}{0, 0.6, 0}
\definecolor{codegray}{rgb}{0.5, 0.5, 0.5}
\definecolor{codepurple}{rgb}{0.58, 0, 0.82}
\definecolor{backcolour}{rgb}{0.95, 0.95, 0.92}

% Esilo de enlaces
\hypersetup{
	colorlinks=true,
	linkcolor=greenPortada,
	filecolor=greenPortada,
	urlcolor=greenPortada,
	pdftitle={Overleaf Example},
	pdfpagemode=FullScreen,
}

\urlstyle{same}

% Estilo de bloques de código
\lstdefinestyle{mystyle}{
	backgroundcolor=\color{backcolour},
	commentstyle=\color{codegreen},
	keywordstyle=\color{magenta},
	numberstyle=\tiny\color{codegray},
	stringstyle=\color{codepurple},
	basicstyle=\ttfamily\footnotesize,
	breakatwhitespace=false,
	breaklines=true,
	captionpos=b,
	keepspaces=true,
	numbers=left,
	numbersep=5pt,
	showspaces=false,
	showstringspaces=false,
	showtabs=false,
	tabsize=2
}

\lstset{style=mystyle}

% Declaración de variables
\newcommand{\logoUdg}{../../../../attachments/images/portada-udg.jpeg}
\newcommand{\logoCucei}{../../../../attachments/images/portada-cucei.jpeg}

% Datos de la Materia
\newcommand{\materia}{Sistemas Operativos}
\newcommand{\theTitle}{9. Glosario}
\newcommand{\profesor}{Violeta del Rocío Becerra Velázquez}
\newcommand{\theAuthor}{Alan Yahir Juárez Rubio}
\newcommand{\code}{218517809}
\newcommand{\carrera}{Ingeniería en Cómputo}
\newcommand{\seccion}{D04}
\newcommand{\startDate}{19 de mayo de 2024}

% Espaciado
\newcommand{\nl}{\par\vspace{0.4cm}}

% Adicionales
\addto\captionsspanish{\renewcommand{\contentsname}{Índice}}
\setlength{\headheight}{60pt}
\pagestyle{fancy}

% Encabezado
\fancyhf{}
\lhead{
	\begin{minipage}[c][2cm][c]{1.3cm}
		\begin{flushleft}
			\includegraphics[width=5cm, height=1.4cm, keepaspectratio]{\logoUdg}
		\end{flushleft}
	\end{minipage}
	\begin{minipage}[c][2cm][c]{0.5\textwidth} % Adjust the height as needed
		\begin{flushleft}	
		{\ifodd\value{page}\materia\fi}
		\end{flushleft}
	\end{minipage}
}
\rhead{
		\begin{minipage}[c][2cm][c]{0.4\textwidth} % Adjust the height as needed
			\begin{flushright}
				{\ifodd\value{page}\else\theTitle\fi}
			\end{flushright}
		\end{minipage}
		\begin{minipage}[c][2cm][c]{1.3cm}
			\begin{flushright}
				\includegraphics[width=5cm, height=1.4cm, keepaspectratio]{\logoCucei}
			\end{flushright}
		\end{minipage}
}

% Pie de página
\rfoot{\ifodd\value{page}\theAuthor\fi}

\renewcommand{\headrulewidth}{3pt}
\renewcommand{\headrule}{\hbox to\headwidth{\color{greenPortada}\leaders\hrule height \headrulewidth\hfill}}

\renewcommand{\lstlistingname}{Código} % Para cambiar el caption de los código
%\renewcommand\thepage{\ifodd\value{page}\else\arabic{page}\fi}

\title{\theTitle}
\author{\theAuthor}

% Comienzo del Documento
\begin{document}
\cfoot{\ifodd\value{page}\else\thepage\fi} % Paginación
\setstretch{1}

\begin{titlepage}
	\centering
	{\huge\textbf{Universidad de Guadalajara}}\par\vspace{0.6cm}
	{\LARGE{Centro Universitario de Ciencias Exactas e Ingenierías}}\vfill
	
	\begin{figure}[h]
		\begin{minipage}[t]{0.45\textwidth}
			\centering
			\includegraphics[width=130px, height=300px, keepaspectratio]{\logoUdg}
		\end{minipage}
		\hfill
		\begin{minipage}[t]{0.45\textwidth}
			\centering
			\includegraphics[width=130px, height=300px, keepaspectratio]{\logoCucei}
		\end{minipage}
	\end{figure}\vfill
	
	{\Large{División de Tecnologías para la Integración CiberHumana}}\vfill
	{\Large\textbf{\materia}}\vfill
	\begin{figure}[h]
		\centering
		\begin{minipage}[t]{0.75\textwidth}
			{\Large
				\textbf{Profesor:} \profesor\nl
				\textbf{Alumno:} \theAuthor\nl
				\textbf{Código:} \code\nl
				\textbf{Carrera:} \carrera\nl
				\textbf{Sección:} \seccion
			}
		\end{minipage}
	\end{figure}\vfill
	{\LARGE{\textbf{\theTitle}}}\vfill
	
	\begin{tcolorbox}[colback=red!5!white, colframe=red!75!black]
		\centering
		Este documento contiene información sensible.\\
		No debería ser impreso o compartido con terceras entidades.
	\end{tcolorbox}\vfill
	{\large \startDate}\par
\end{titlepage}

% Índices
\clearpage
\tableofcontents

%\clearpage
%\listoffigures
	
%\clearpage
%\listoftables
% Fin de Índices

\clearpage
\section{Glosario}

\subsection{Fundamentos de Sistemas Operativos}

\begin{enumerate}
	\item \textbf{Sitema Operativo}: Conjunto de programas los cuales permiten al sistema poder controlar y gestionar los recursos del sistema para los diferentes servicios/programas puedan hacer uso de estos recursos y, en base a ello, que la computadora ejecute instrucciones, tareas. En otras palabras el sistema operativo es el intermediario, el puente de comunicación entre el usuario y el computador.
\end{enumerate}

\subsubsection{Procesamiento}

\begin{enumerate}[resume*]
	\item \textbf{Serie}: Ejecución de tareas secuencialmente. Esto significa que una tarea debe completarse antes de que comience la siguiente. Es como seguir una lista de tareas en la que cada tarea depende del resultado de la anterior. Este enfoque puede ser menos eficiente en términos de tiempo de ejecución, ya que las tareas se ejecutan de manera secuencial, lo que puede llevar más tiempo para completar todas las tareas en comparación con el procesamiento en paralelo.

	\item \textbf{Paralelo}: Ejecución de tareas simultáneamente. Se utilizan recursos de hardware múltiples o compartidos. Esto permite que varias tareas se realicen al mismo tiempo, lo que puede mejorar significativamente el tiempo de ejecución total de las tareas. El procesamiento en paralelo puede tomar varias formas, como el procesamiento simultáneo en múltiples núcleos de CPU, la ejecución de múltiples hilos de un proceso en paralelo o la distribución de tareas entre varios nodos en un sistema distribuido.

	\item \textbf{Por Lotes}: Ejecución de un conjunto de trabajos (también llamados procesos) de manera secuencial o simultánea, agrupados en lotes o grupos.
\end{enumerate}

\subsubsection{Modos de Operación}

\begin{enumerate}[resume*]
	\item \textbf{\textit{Kernel Mode}}: Se tiene acceso completo a todo el hardware y puede ejecutar cualquier instrucción que la máquina sea capaz de ejecutar.
	
	\item \textbf{\textit{User Mode}}: Se tiene acceso a solo un subconjunto de las instrucciones de la máquina. Por lo general, la instrucciones que afectan al control de la máquina o hacen instrucciones I/O son inaccesibles por los programas de modo de usuario
\end{enumerate}

\subsection{Modelos de Sistemas Operativos}

\begin{enumerate}[resume*]
	\item \textbf{Monlítico}: Consiste en que el SO corra como un solo programa en \textit{Kernel Mode}. El SO es escrito como una colección de procedimientos, enlazados entre sí en un único programa binario ejecutable grande.

	\item \textbf{Cliente-Servidor}: Variación pequeña de la idea de microkernel en la que se distinguen dos clases de procesos:

	\begin{enumerate}[resume*]
		\item \textbf{Servidores}: Aquellos que proveen algún servicio.
		\item \textbf{Clientes}: Aquellos que utilizan los anteriores.
	\end{enumerate}

	\item \textbf{Máquina Virtual}: conjunto de dispositivos de hardware emulados en el cual corre el SO. Para el SO le es indistinguible una máquina física de una virtual, lo cual permite que este funcione de la misma manera.
\end{enumerate}

\subsection{Procesos}

\begin{enumerate}[resume*]
	\item \textbf{\textit{Foreground}}: Aquellos procesos que, al ser ejecutados, son visibles para el usuario. El usuario interactúa directamente con ellos. P. ej: reproducción de video, uso de procesadores de texto (p. ej. word,), reproducir un videojuego, etc.
	
	\item \textbf{\textit{Background}}: Aquellos que, al ser ejecutados, no son visibles para el usuario. El usuario interactúa indirectamente o inclusive no interactúa con ellos. P. ej: mantenimiento del sistema (virus, limpieza, actualización...), descarga de archivos, \textit{deamons} (p. ej: servicios del sistema), etc.
\end{enumerate}

\subsection{Políticas de Planificación}

\begin{enumerate}[resume*]
	\item \textbf{Políticas Apropiativas}: El proceso en ejecución puede ser interrumpido y pasar a estado listo. Esta decisión la puede tomar el sistema operativo cuando llega un nuevo proceso, cuando un proceso bloqueado pasa a estado listo, cuando se activa un proceso suspendido, o periódicamente al producirse una interrupción del reloj. 
	
	\item \textbf{Políticas no Apropiativas}: Cuando un proceso empieza a ejecutarse, solo se le puede retirar la CPU si se bloquea o termina.
	
	\item El \textbf{\textit{Quantum}}: es un intérvalo de tiempo ya establecido que se utiliza en algorimtos de planificación tal como el round robin o el round robin mejorado.
\end{enumerate}

\subsection{Hilos}

\begin{enumerate}[resume*]
	\item \textbf{Proceso}: Instancia de un programa en ejecución. Está conformado por el contador del programa, registros y variables.
	
	\item \textbf{Hilo}: Conocido también como proceso ligero. Es una unidad básica de uso del CPU. Un hilo se encuentra dentro de un proceso. Un proceso puede contener un hilo (monohilo) o múltiples hilos (multihilo). Está conformado por un contador de programa, un conjunto de registros y un espacio en stack
\end{enumerate}

\subsection{Interbloqueo}

\begin{enumerate}[resume*]
	\item \textbf{Interbloqueo}: Situación en la que un proceso $A$ necesita un recurso ocupado por un proceso $B$ para continuar y, seguidamente, el proceso $B$ requiere un recurso ocupado por el proceso $A$. En otras palabras, es cuando dos o más procesos no pueden continuar debido a que cada proceso involucrado requiere un recurso ocupado por otro proceso para continuar con su ejecución.
	
	\item \textbf{Bloqueo Mutuo}: Situación en la que dos o más procesos en un sistema operativo quedan atrapados en un ciclo continuo de respuesta a las acciones del otro, pero sin realizar ningún progreso real en su ejecución. Esto puede ocurrir cuando dos o más procesos intentan ajustar sus acciones en respuesta a las acciones del otro, pero ninguna de las acciones resulta en un avance en la ejecución de ninguno de los procesos.
\end{enumerate}

\subsection{Archivos y Directorios}

\begin{enumerate}[resume*]
	\item \textbf{Sistema de archivos}: Sistema que el SO utiliza para el manejo de archivos: como están estructurados, como son nombrados, accedidos, usados, protegidos, implementados y manejados.
	
	\item \textbf{Archivos}: Mecanismo de abstracción que proporcionan una forma de almacenar información en dispositivos de almacenamiento secundario.
	
	\item \textbf{Directorios}: Archivos que contienen información sobre otros archivos y se diferencian de ellos en el modo de acceso.
\end{enumerate}
% Referencias

\subsection{Almacenamiento y Gestión de Archivos}

\begin{enumerate}[resume*]
	\item \textbf{Índices}: en SO son estructuras de datos esenciales para optimizar la búsqueda y acceso a información en el sistema de archivos. Su función principal es mejorar el rendimiento de las operaciones de lectura y escritura al permitir la ubicación rápida de archivos y bloques de datos en el almacenamiento.
	
	\item \textbf{Dispersión}: También conocida como hashing. Es una técnica fundamental en SO para distribuir elementos en una estructura de datos de manera uniforme. Su objetivo principal es minimizar las colisiones y mejorar el rendimiento de las operaciones de búsqueda, inserción y eliminación.
\end{enumerate}

\subsection{Criptografía}

\begin{enumerate}[resume*]
	\item \textbf{Criptografía}: Técnica que permite proveer seguridad a la información y a la comunicación. Esta consiste en codificar el contenido de un archivo o de un texto de tal manera que solamente pueda desencriptarlo el o los usuarios a los que están destinada esta información.
	
	\item \textbf{Criptografía Símétrica}: Algoritmo en donde el emisor y el receptor de un mensaje usan solamente una llave para encriptar y desencriptar mensajes. Este es rápido y simple, sin embargo, el problema reside en que el emisor y el receptor tienen que intercambiar llaves de forma segura.
	
	\item \textbf{Criptografía Asimétrica}: A diferencia del simétrico, este sistema utiliza una llave para encriptar la información (llave pública) y otra llave para descencriptarla (llave privada). El emisor utiliza la \textbf{llave pública} del destinatario para encriptar la información y el receptor utiliza su \textbf{llave privada} para desencriptar dicho mensaje.
	
	\item \textbf{Funciones Hash}: En este algoritmo no se utilizan llaves. Se calcula un valor \textit{hash} con una longitud fija según el texto sin formato. Esto hace imposible recuperar el contenido del texto sin formato.
\end{enumerate}
\nocite{*} % Para incluir todas las referencias sin necesidad de citarlas

\subsection{Esteganografía}

\begin{enumerate}[resume*]
	\item \textbf{Esteganografía}: Arte y la ciencia de ocultar mensajes dentro de otros mensajes o archivos de manera que el mensaje oculto no sea aparente para un observador casual.
	
	\item \textbf{Enmascaramiento}: En este caso la información se oculta dentro de una imagen digital usando marcas de agua donde se introduce información, como el derecho de autor, la propiedad o licencias. El objetivo es diferente de la esteganografía tradicional, lo que se pretende es añadir un atributo a la imagen que actúa como cubierta. De este modo se amplía la cantidad de información presentada.
	
	\item \textbf{Compresión de Datos}: Esta técnica oculta datos basados en funciones matemáticas que se utilizan a menudo en algoritmos de la compresión de datos. La idea de este método es ocultar el mensaje en los bits de datos menos importantes.
	
	\item \textbf{Métodos de Sustitución}: Una de las formas más comunes de hacer esto es alterando el bit menos significativo (LSB). En archivos de imagen, audio y otros, los últimos bits de información en un byte no son necesariamente tan importantes como los iniciales. Por ejemplo, 10010010 podría ser un tono de azul. Si solo cambiamos los dos últimos bits a 10010001, podría ser un tono de azul que es casi exactamente igual. Esto significa que podemos ocultar nuestros datos secretos en los dos últimos bits de cada píxel de una imagen, sin cambiar la imagen de forma notable. Si cambiamos los primeros bits, lo alteraría significativamente.
\end{enumerate}

\clearpage
\section{Conceptos publicados}

\begin{figure}[h]
	\centering
	\includegraphics[width=0.65\textwidth]{./img/img-1.jpg}
\end{figure}

\begin{figure}[h]
	\centering
	\includegraphics[width=0.65\textwidth]{./img/img-2.jpg}
\end{figure}

\begin{figure}[h]
	\centering
	\includegraphics[width=0.65\textwidth]{./img/img-3.jpg}
\end{figure}

\end{document}
